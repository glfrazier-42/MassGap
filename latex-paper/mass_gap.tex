\documentclass[letterpaper,11pt,leqno]{article}
\usepackage{paper}
\bibliographystyle{planar_gravity}
\usepackage{dcolumn}
\usepackage{booktabs}
\newcolumntype{d}[1]{D{.}{.}{#1}}

% Paper title for PDF metadata:
% Planar Gravity: A Geometric Explanation for Dark Matter Effects in Rotating Galaxies
\hypersetup{pdftitle={
Planar Gravity and Flat Rotation Curves in Rotating Galaxies
}}

% Path to BibTeX file:
\newcommand{\bib}{planar_gravity.bib}

% Path to figures PDF:
\newcommand{\pdf}{figures.pdf}

\begin{document}

% \title{Planar Gravity: A Geometric Explanation for Dark Matter Effects in Rotating Galaxies}
\title{Planar Gravity and Flat Rotation Curves in Rotating Galaxies}

\author{Gregory L. Frazier \thanks{Independent researcher. The author
    thanks Claude (Anthropic) for assistance with analysis and
    manuscript preparation.}}

\date{December 2025}   

\begin{titlepage}
\maketitle

In rotating disk galaxies, observed rotation velocities exceed
predictions from visible matter alone, a phenomenon attributed to
\emph{dark matter}. We instead propose that it arises from a
previously unrecognized property of frame-dragging: a weak planar
gravitational component generated by spinning objects, confined to the
equatorial plane and scaling as $1/r$ rather than the $1/r^3$
predicted by the Kerr effect or the $1/r^2$ drop-off of Newtonian
gravity. When spin axes align perpendicular to the galactic
plane---occuring dynamically as emergent behavior in disk
galaxies---the planar fields act coherently, producing flat rotation
curves at galaxy edges. We validate this model through two independent
lines of evidence. First, pulsar observations reveal
location-dependent spin-down: pulsars near the galactic plane
decelerate up to ten times faster than those at high $|z|$ ($\rho =
-0.38$, $p < 10^{-43}$), consistent with gravitational torquing but
unexplainable by standard magnetic dipole braking. Second, the model
reproduces SPARC rotation curves with median residuals of 0.108~dex
and zero free parameters per galaxy, matching the empirical Radial
Acceleration Relation's precision while deriving from physical
principles rather than curve-fitting. Third, we present a
discriminating test: the Milky Way vertical velocity gradient. Planar
gravity predicts that stars at greater height $|z|$ above the disk
should orbit more slowly at fixed galactocentric radius; spherical
dark matter halos predict no such dependence. Gaia DR3 data confirm
the gradient exists, with $V_\phi$ dropping from $\sim$220~km/s at $|z| =
0$ to $\sim$150--180~km/s at $|z|=4.5$~kpc for $R < 15$~kpc.  The planar
gravity model explains why the effect appears in rotating disks but
not pressure-supported systems: morphology, not mass, determines
whether coherent spin alignment can develop. Our central claim is that
flat rotation curves arise from coherently spinning baryonic matter
rather than dark matter particles.

\end{titlepage}

%==============================================================================
\section{Introduction}\label{s:introduction}
%==============================================================================

In rotating disk galaxies, stars at the galactic periphery orbit
faster than predicted by the visible matter distribution. This ``flat
rotation curve'' phenomenon, first systematically documented by
\citet{Rubin1980}, has been confirmed in thousands of galaxies and
represents one of the most robust observations in extragalactic
astronomy. The prevailing explanation---that galaxies are embedded in
extended halos of non-baryonic dark matter---has become
central to modern cosmology despite the continued absence of direct
particle detection. A conundrum of this phenomenon is its
variability. Whether a galaxy exhibits flat rotation curves depends on
both its mass and its morphology: the effect appears fully in
late-type spirals and disk-dominated systems, weakly or not at all in
elliptical galaxies and pressure-supported systems. There is no
obvious reason, under the dark matter hypothesis, for this pattern.

\citet{McGaugh2016} performed a comprehensive analysis using 153 disk
galaxies from the SPARC database. They examined the relationship
between $g_{\text{bar}}$, the centripetal acceleration predicted from
baryonic matter using Newtonian gravity, and $g_{\text{obs}}$, the
acceleration derived from observed rotation velocities. They
discovered an extraordinarily tight empirical relation---the Radial
Acceleration Relation (RAR):
\begin{equation}
g_{\text{obs}} = \frac{g_{\text{bar}}}{1 - \exp\left(-\sqrt{g_{\text{bar}}/g^\dagger}\right)}
\label{e:rar}
\end{equation}
where $g^\dagger = 1.20 \pm 0.02 \times 10^{-10}$~m/s$^2$ is a
characteristic acceleration scale.  The RAR exhibits remarkable
properties: universality across galaxies spanning three orders of
magnitude in mass; scatter of only 0.13~dex (which is consistent with
measurement uncertainty); no free parameters per galaxy; and validity
from galaxy centers (where Newtonian gravity dominates) through outer
regions (where the dark matter effect dominates).

The existence of the RAR poses a fundamental challenge to the dark
matter paradigm. Under $\Lambda$CDM cosmology
\citep{Riess1998,Perlmutter1999}, dark matter halos form first from
primordial density fluctuations, baryons subsequently fall into these
potential wells, and the two components evolve somewhat
independently. There is no inherent reason for the dark matter
distribution to track the baryon distribution with such extraordinary
precision at every radius in every rotating
galaxy. \citet{McGaugh2016} identified three possible interpretations:
1)\,galaxy formation processes somehow tune dark matter to track
baryons; 2)\,dark matter responds directly to baryons through unknown
coupling; or 3)\,the phenomena attributed to dark matter reflect
incomplete understanding of gravity.

This paper explores the third interpretation. We propose that the flat
rotation curves are caused by a previously unrecognized property of
\emph{frame-dragging}. General relativity predicts that rotating
masses produce gravitomagnetic effects (frame-dragging)
\citep{LenseThirring1918}, which has been experimentally confirmed by
Gravity Probe B \citep{GravityProbeB2011}. Observations suggest that
frame-dragging also produces a weaker two-dimensional gravitational
component, confined to the plane perpendicular to the rotation axis,
falling off as $1/r$ but otherwise has the properties of
frame-dragging (exerting torque on other spinning objects and shear
force on moving objects, perpendicular to their movement vector). We
refer to this as \emph{planar gravity}. Three key features of planar
gravity enable the flat rotation curves observed at the edges of
rotating galaxies:
\begin{enumerate}
  \item Spinning objects exert torque on each others' axis of rotation,
    such that each is moved towards the other's orientation;
  \item In rotating galaxies, the resulting, emergent
    self-organization creates a dynamic alignment such that some
    fraction of the spinning objects have their axes perpendicular to
    the galactic plane via continuous mutual torquing; and
  \item The $1/r$ falloff of planar gravity combined with the
    alignment of spin axes allows planar gravity to dominate at large
    radii.
\end{enumerate}

When considering individual objects, planar gravity is undetectable:
its strength is approximately $3 \times 10^{-21}$ that of Newtonian
gravity. However, when a given fraction of spinning objects' rotation
axes align perpendicular to the galactic plane---as occurs
dynamically in disk galaxies---their planar fields add
coherently. This explains both the morphology dependence and
why the effect appears at galactic but not smaller scales. We validate
this hypothesis through two independent lines of evidence:

\begin{itemize}
\item Pulsar observations demonstrate the mechanism at the individual-object
level. Analysis of over 1,200 pulsars reveals that spin-down rates correlate
strongly with distance from the galactic plane ($\rho = -0.403$, 
$p = 7.59^{-48}$), consistent with gravitational torquing extracting rotational
energy to reorient spin axes. This location-dependent braking cannot be
explained by standard magnetic dipole radiation. A smaller sample of 20
pulsars with measured spin axes shows suggestive evidence for progressive
alignment toward perpendicularity.

\item Galaxy rotation curves demonstrate the collective effect. Applying a
mass-dependent meta-function to predict alignment efficiency, the model
reproduces 175 SPARC rotation curves with median RMS residual of 0.108~dex
and zero free parameters per galaxy---accuracy matching the empirical
Radial Acceleration Relation while deriving from physical mechanisms rather
than phenomenological curve-fitting.

\item The Milky Way vertical velocity gradient provides a discriminating test.
Planar gravity predicts that circular velocity decreases with height above
the disk; spherical dark matter halos predict no such dependence. Gaia DR3
data show precisely the predicted gradient, independently confirmed by
thin-disk models that match observations better than halo models.

\end{itemize}

The remaineder of this paper is structured as
follows. Section~\ref{s:background} reviews existing explanations
including particle dark matter and MOND. Section~\ref{s:mechanism}
develops the physical basis for planar gravity as a hitherto
undescribed component of frame-dragging. Section~\ref{s:emergent}
models the emergent behavior that planar gravity might elicit in a
rotating galaxy and makes predictions that are then tested in
subsequent sections.  Section~\ref{s:pulsars} tests predictions
regarding the behavior of pulsars (individually and in aggraget) in
the presence of planar gravity, and Sec.~\ref{s:sparc_validation}
validates the planar model against the SPARC
dataset. Section~\ref{s:mw_vertical} presents a discriminating test
using the Milky Way vertical velocity gradient. We end with a
discussion of limitations and a summary.


%==============================================================================
\section{Background: Observations and Existing Frameworks}\label{s:background}
%==============================================================================

% \subsection{Historical Context}

The missing mass problem was first identified by \citet{Zwicky1933}
through observations of the Coma Cluster, where galaxy velocities
implied more mass than visible stars could provide. The phenomenon
became undeniable with \citet{Rubin1980}'s systematic observations of
spiral galaxy rotation curves, finding that rotation velocities remain
approximately constant at large radii rather than declining as
Keplerian dynamics predict.

% \subsection{Particle Dark Matter}

The prevailing explanation proposes that galaxies are embedded in
extended halos of non-baryonic particles that interact only
gravitationally. The dark matter distribution within halos is
characterized by density profiles fitted to observations: the
pseudo-isothermal profile assumes flat central density; the NFW
profile \citep{NFW1996} predicts density rising steeply toward
centers; the Burkert profile \citep{Burkert1995} was proposed in
response to NFW's failure to match observed cores in dwarf galaxies.

%% The need for multiple competing profiles illustrates a fundamental
%% limitation: particle dark matter lacks a universal predictive
%% framework for galactic dynamics. As \citet{McGaugh2016} observed,
%% ``such models are unnecessary. The distribution of dark matter follows
%% directly from the relation, and can be written entirely in terms of
%% the baryons.'' Fundamentally, particle dark matter is post-hoc:
%% rotation curves are measured, baryonic contributions subtracted, and
%% halo profiles fitted to explain the residual.

% \subsection{Modified Newtonian Dynamics}

An alternate approach is the \citet{Milgrom1983} proposal that the
behavior of gravity changes at low accelerations. Below a
characteristic scale $a_0 \approx 1.2 \times 10^{-10}$~m/s$^2$,
gravitational dynamics transition from Newtonian behavior to a
different regime. MOND's predictive success at galactic scales is
genuine---the theory anticipated the RAR decades before its empirical
discovery. However, its purely phenomenological character limits its
explanatory power: it modifies equations without physical motivation.

% \subsection{Scope of This Work}

This paper proposes a mechanism operating that naturally produces the
RAR's functional form and explains morphology dependence. We focus
specifically on rotating disk galaxies, the systems for which the RAR
is established.

%==============================================================================
\section{Planar Gravity: The Physical Mechanism}\label{s:mechanism}
%==============================================================================

Before describing the proposed mechanism, we explain the reasoning
that led to it. The dark matter effect in galaxies displays the
hallmarks of emergent behavior: it appears abruptly above a threshold
of approximately $10^8$--$10^9$ stars; it depends on morphology
(appearing in rotating disks but not pressure-supported spheroids of
similar mass); and it correlates with baryonic mass so tightly that
the Radial Acceleration Relation holds across three orders of
magnitude with scatter consistent with measurement uncertainty. These
properties suggest collective self-organization rather than an
independent substance.

Emergent behavior requires a coupling mechanism between individual
objects. Angular momentum is the natural candidate. Stars in a disk
share a common orbital plane; if their spin axes also tend toward a
common orientation, any gravitational effect of spin would add
coherently. In a spheroidal system with random orientations, the same
effect would cancel. This matches the observed morphology dependence.

If aligned spin produces any collective gravitational component, the
phenomenology constrains its form. Flat rotation curves require
gravitational acceleration falling as $1/r$ at large radii—precisely
what emerges from a field confined to propagate in two dimensions
rather than three. We therefore propose that spinning masses generate
a weak planar gravitational component perpendicular to their spin
axes, falling off as $1/r$. Whether General Relativity predicts such a
component for configurations of coherently aligned angular momentum
is, to our knowledge, an unexplored question; the nonlinearity of the
field equations means that the collective field of $10^{11}$ aligned
rotators cannot be derived by simply summing individual Lense-Thirring
solutions. We proceed phenomenologically: the observations demand an
effect with these properties, and we test whether the hypothesis is
consistent with independent data.

General relativity establishes that
rotating masses affect spacetime geometry differently than static
masses. The Kerr metric \citep{Kerr1963} describes the spacetime
around a rotating body, and in the weak-field limit reduces to the
Lense-Thirring solution \citep{LenseThirring1918}. A rotating mass
``drags'' the surrounding spacetime in the direction of its rotation,
an effect directly analogous to how moving electric charges generate
magnetic fields.

This gravitomagnetic field produces two distinct effects on nearby
objects. A moving test particle experiences a Lorentz-like force
perpendicular to its velocity:
\begin{equation}
\vec{a} = -2\vec{v} \times \vec{B}_g
\label{e:lorentz}
\end{equation}
where $\vec{B}_g$ is the gravitomagnetic field. A spinning test body
experiences a torque that causes its spin axis to precess:
\begin{equation}
\frac{d\vec{S}}{dt} = \vec{\Omega} \times \vec{S}
\label{e:precession}
\end{equation}
where the precession rate depends on the gravitomagnetic field
strength. In both cases, Newton's third law applies: the torque or
deflection experienced by a test body produces an equal and opposite
effect on the source.

The gravitomagnetic field from a body with angular momentum $\vec{J}$
has a dipole structure---strongest in the equatorial plane,
vanishing along the spin axis---and falls off as $1/r^3$:
\begin{equation}
\vec{B}_g = \frac{G}{c^2 r^3}\left[3(\vec{J} \cdot \hat{r})\hat{r} - \vec{J}\right]
\label{e:gravitomagnetic}
\end{equation}
This rapid falloff, combined with the factor of $c^2$ in the
denominator, renders frame-dragging extraordinarily weak. For Earth,
the Lense-Thirring precession is approximately 39~milliarcseconds per
year---some 170 times smaller than the geodetic precession caused by
spacetime curvature alone, and roughly $10^{-9}$ of Earth's rotation
rate. Gravity Probe B confirmed this prediction to within 20\%
\citep{GravityProbeB2011}. At astronomical scales, frame-dragging is
negligible except near compact objects such as neutron
stars and black holes.

Specifically, we propose that spinning masses generate torque and shear forces not
only as described above (acting in three dimensions, falling off as
$1/r^3$), but also in the two-dimensional plane perpendicular to the
spin axis. This planar component is far weaker than the Kerr-metric
gravitomagnetic field, but possibly because it is confined to the equatorial
plane, it falls off as $1/r$ rather than $1/r^3$. That said, we are inferring the
existence of planar gravity via phenomena. Its properties are those
of frame-dragging:
\begin{itemize}
\item Generated by angular momentum;
\item Creates both torque and shear forces; and
\item Weak relative to Newtonian gravity.
\end{itemize}
Thus, we hypothesize that planar gravity \textit{extends the physics 
of frame-dragging} with a two-dimensional geometric component, but we 
do not attempt to derive planar gravity from GR. Our focus here is on 
phenomenological evidence that planar gravity exists and produces 
the observed flat rotation curves.


\section{A Prediction of Emergent Alignment in Disk Galaxies}\label{s:emergent}

\emph{Complex systems} are systems comprised of large numbers of
(relatively) simple objects whose localized interactions create
\emph{emergent behavior} \citep{Anderson1972}. Planar gravity
produces torques on spinning objects---when spinning object $B$ is in
the equatorial plane of spinning object $A$, both objects experience
torque on their spin axes, pulling each axis towards the other. In a
spherical galaxy, where there is no coherent orbital plane, this
torquing does not result in any form of organization. However, in
rotating galaxies, where the vast majority of the spinning objects
orbit in a narrow range of inclinations, there is an opportunity for
self-organization to occur.

Emergent behavior is notoriously difficult to predict; it does not
lend itself to closed-form solutions.  To build a prediction of the
emergent behavior that would result from planar gravity acting in a
rotating galaxy, we employ a mean-field approach. An N-body simulation
would have required specifying microscopic parameters---the strength
of planar gravity coupling, distributions of spin rates,
gyroscopic stability coefficients---that are not yet constrained. With
incorrect parameters, such simulations might fail to exhibit emergence
even if the underlying physics is real, or produce spurious effects
that mislead interpretation. Mean-field theory sidesteps this
difficulty by working at the level of effective parameters whose
ratios determine qualitative behavior, revealing phase structure and
equilibrium states without requiring precise microscopic knowledge.

The mean-field approach treats the alignment
state of the stellar population statistically, replacing individual
object-to-object interactions with an aggregate field that represents
the collective influence of all aligned objects. Let $f$ represent
the alignment fraction---the fraction of objects whose spin axes are
perpendicular to the galactic orbital plane. The evolution of $f$ is
governed by two competing processes:

\begin{figure}[t]
\subcaptionbox{Alignment Evolution for Various Disturbance Strengths, $\alpha=1.0$\label{f:varying_beta}}{
  \includegraphics[scale=0.4]{figures/beta_comparison_paper.png}
}\hfill
\subcaptionbox{Alignment Evolution for Various Alignment Strengths,
  $\beta=0.5$\label{f:varying_alpha}}{
  \includegraphics[scale=0.4]{figures/alpha_comparison_paper.png}
}
\caption{Emergent Behavior Prediction from Mean Field Analysis}
\note{When disturbance increases, e.g., due to increased galactic
  plane thickness, alignment probability is diminished. When alignment
  strength increases, e.g., due to more spinning
  objects in a shared orbital plane, alignment probability increases.}
\label{f:emergece_predictions}
\end{figure}

\begin{figure}[h]
  \begin{center}
    \includegraphics[scale=0.6]{figures/radial_alignment_profile_paper.png}
  \end{center}
\caption{Diminishing Disturbance as Radius Increases}
\note{Extending the emergent behavior prediction to the galactic
  disk---as the disk thins with radius, the prediction that disturbance
  drops and alignment correspondingly grows.}
\label{f:radial_beta}
\end{figure}

\begin{enumerate}

\item Alignment coupling ($\alpha$): Objects experience
  torque from the coherent planar gravity field of already-aligned
  objects, driving unaligned objects toward alignment. This produces a
  term proportional to $f(1-f)$---alignment is self-reinforcing
  (proportional to $f$) but saturates as the unaligned population
  ($1-f$) is depleted.

\item Perturbation ($\beta$): Stellar encounters, tidal
  interactions, and passage through molecular clouds randomly reorient
  spin axes, driving aligned objects back toward random
  orientations. This produces a term proportional to $f$.
\end{enumerate}

The resulting dynamical equation is:
\begin{equation}
\frac{df}{dt} = \alpha \cdot f \cdot (1-f) - \beta \cdot f
\label{e:mean_field}
\end{equation}
At equilibrium ($df/dt=0$), this yields:
\begin{equation}
f^* = \begin{cases}
0 & \text{if } \alpha \leq \beta \\
1 - \beta/\alpha & \text{if } \alpha > \beta
\end{cases}
\label{e:equilibrium}
\end{equation}
This reveals a \emph{phase transition} at $\alpha = \beta$. When
perturbations dominate ($\beta > \alpha$), no coherent alignment
develops regardless of time---the system remains in a disordered
state. When coupling dominates ($\alpha > \beta$), the system evolves
toward a stable aligned fraction that increases as the ratio
$\alpha / \beta$ grows.

The mean field analysis offers the key insight that $\alpha$ is strongest at
the galactic plane, where rotating objects share an orbital plane and
so are "bathed" in each other's planar gravity, and $\beta$'s influence is
proportional to depth. Dense galactic cores experience frequent
perturbations (high $\beta$), suppressing alignment. Sparse
environments—dwarf galaxies, galactic peripheries—experience few
perturbations (low $\beta$), allowing especially strong alignment to develop
and persist.



This mean-field framework generates testable predictions at multiple scales:

\begin{enumerate}
\item \emph{Individual dynamics}: Spinning objects should progressively 
align toward perpendicularity, with alignment rate increasing near the 
galactic plane where coupling dominates perturbations. This alignment should 
extract rotational energy, producing location-dependent spin-down.

\item \emph{Radial dependence}: Alignment efficiency should increase 
toward galaxy edges where stellar density (and thus $\beta$) decreases, 
predicting stronger planar gravity per unit mass at large radii.

\item \emph{Mass scaling}: More massive galaxies should exhibit lower 
overall alignment due to uniformly higher encounter rates, predicting an 
inverse correlation between galaxy mass and planar gravity efficiency.

\item \emph{Morphological selection}: The effect should appear in 
rotating disks where coherent orbital planes enable constructive alignment 
coupling, but be absent in pressure-supported spheroids where random 
encounter directions prevent coherent alignment.
\end{enumerate}

The following sections test these predictions against pulsar observations 
and galaxy rotation curves.

%% \begin{figure}[t]
%% \includegraphics[scale=0.4,page=7]{\pdf}
%% \caption{Reenforced alignment} \note{In a flat galaxy, spinning
%%   stellar objects whose orbital planes heavily overlap self-organize
%%   their spin axes perpendicular to te orbital plane.}
%% \label{f:alignment_cartoon}\end{figure}


\section{Pulsar Spin Axis Alignment and Gravitational Self-Organization}\label{s:pulsars}

The emergent behavior framework makes specific, testable predictions
about individual spinning objects in disk galaxies. We now examine
whether pulsar observations confirm these predictions.  We examine two
complementary datasets. First, a small sample of 20 pulsars from
\citet{Johnston2005} for which spin axis orientations have been
directly measured, showing a statistically suggestive excess of
perpendicular alignments correlated with both proximity to the
galactic plane and age. Second, a large sample of over 1,200 pulsars
from \citet{Johnston2023} for which precise spin periods $P$ and
spin-down rates $\dot{P}$ are available, revealing a strong
correlation between the rate of deceleration and distance from the
galactic plane.

Together, these datasets demonstrate that: (a) spinning objects in the
galactic disk tend toward perpendicular spin axis alignment, (b) this
tendency increases with proximity to the galactic plane, and (c) the
energy driving this alignment is extracted from the objects'
rotational kinetic energy.

\subsection{The Alignment Mechanism: Spin Axis Observations}

\citet{Johnston2005} measured proper motions and polarization angles
for 20 pulsars, allowing determination of both velocity vectors and
spin axis orientations projected onto the plane of the sky. Their
analysis sought correlations between spin and velocity directions as a
test of supernova kick mechanisms. They found that older pulsars
lost the correlation between spin and velocity; they attributed this
to changes in the pulsars' motion vectors over time.

% \usepackage{dcolumn}
% \usepackage{booktabs}
% \newcolumntype{d}[1]{D{.}{.}{#1}}

\begin{table}
\centering
\small
\caption{Johnston 2005 pulsars: spin axis alignment analysis}
\label{t:johnston2005}
\begin{tabular}{
  l
  d{2.2}
  d{2.1}
  d{1.3}
  d{2.1}
  d{3.1}
  d{2.1}
  cc
}
\toprule
Name & \multicolumn{1}{c}{Age} & \multicolumn{1}{c}{$b$} & \multicolumn{1}{c}{$|z|$} & \multicolumn{1}{c}{PA$_0$} & \multicolumn{1}{c}{PA$_{\text{plane}}$} & \multicolumn{1}{c}{Angle} & Par. & Perp. \\
     & \multicolumn{1}{c}{(Myr)} & \multicolumn{1}{c}{($^\circ$)} & \multicolumn{1}{c}{(kpc)} & \multicolumn{1}{c}{($^\circ$)} & \multicolumn{1}{c}{($^\circ$)} & \multicolumn{1}{c}{($^\circ$)} & Motion & Plane \\
\midrule
J0835-4510 & 0.01  & -2.8  & 0.014 & 36.8  & 143.3 & 73.5 & Y  & -- \\
J0742-2822 & 0.16  & -2.6  & 0.095 & -81.7 & 150.4 & 52.1 & Y  & -- \\
J1453-6413 & 1.00  & -4.4  & 0.161 & -56.9 & 63.2  & 59.9 & Y  & -- \\
J1935+1616 & 1.00  & -1.9  & 0.189 & 10.1  & 29.0  & 18.9 & --  & -- \\
J1709-1640 & 1.58  & 13.8  & 0.190 & 15.0  & 34.3  & 19.3 & Y  & -- \\
J0630-2834 & 2.51  & -16.9 & 0.407 & 26.0  & 157.8 & 48.2 & Y  & -- \\
J1645-0317 & 3.16  & 26.1  & 0.484 & 56.0  & 32.5  & 23.5 & --  & -- \\
J1844+1454 & 3.16  & 8.3   & 0.319 & -52.0 & 26.0  & 78.0 & Y  & -- \\
J1913-0440 & 3.16  & -6.9  & 0.337 & -68.0 & 26.8  & 85.2 & --  & Y \\
J1932+1059 & 3.16  & -3.8  & 0.024 & -11.3 & 28.6  & 39.9 & --  & -- \\
J1136+1551 & 5.01  & 69.2  & 0.094 & -78.0 & 144.1 & 42.1 & --  & -- \\
J1430-6623 & 5.01  & -5.4  & 0.337 & -28.5 & 68.1  & 83.4 & Y  & Y \\
J1740+1311 & 7.94  & 21.7  & 0.555 & -46.0 & 24.3  & 70.3 & Y  & -- \\
J0820-1350 & 10.00 & 12.5  & 0.433 & 65.0  & 147.6 & 82.6 & Y  & Y \\
J0953+0755 & 15.85 & 43.7  & 0.070 & 14.9  & 149.8 & 45.1 & --  & -- \\
J1921+2153 & 15.85 & 3.7   & 0.179 & -35.0 & 27.8  & 62.8 & --  & -- \\
J1239+2453 & 25.12 & 86.4  & 0.858 & -66.0 & 140.6 & 26.6 & Y  & -- \\
J1456-6843 & 39.81 & -8.5  & 0.067 & -31.6 & 62.3  & 86.1 & --  & Y \\
J1900-2600 & 50.12 & -13.3 & 0.460 & -43.0 & 24.0  & 67.0 & --  & -- \\
J0525+1115 & 79.43 & -13.4 & 0.721 & -65.0 & 148.3 & 33.3 & --  & -- \\
\bottomrule
\end{tabular}
\note{Pulsars ordered by characteristic age. ``Par. Motion'' indicates spin 
axis parallel to velocity vector ($\Psi < 10^\circ$ or $> 80^\circ$). 
``Perp. Plane'' indicates spin axis perpendicular to galactic plane 
(PA$_{\text{plane}} > 80^\circ$). Position angle measurements from 
\citet{Johnston2005}.}
\end{table}

We propose an alternative interpretation. The observed
pattern---younger pulsars showing spin-velocity correlation while
older pulsars do not---is equally consistent with a gravitational
torque that reorients spin axes toward a common direction
(perpendicular to the galactic plane) regardless of initial
orientation or velocity direction. As pulsars age, gravitational
torquing would progressively erase any birth-established spin-velocity
correlation.

Table~\ref{t:johnston2005} presents the Johnston 2005 data.
We have added a ``|z|'' column to indicate the distance of the pulsar
from the galactic plane, a column to indicate whether
the spin axis is aligned with its movement vector (from
\citet{Johnston2005}, $\Psi < 10^\circ$
or $\Psi > 80^\circ$), and a column to indicate whether the
spin axis is perpendicular to the galactic plane.

\citet{Johnston2005} noted and explained the limited data set
size---the direction of a pulsar's spin axis can only be identified
under very specific conditions, and these are the twenty pulsars whose
spin axis could be determined at the time the paper was
published. With only 20 pulsars, statistical conclusions cannot be
definitive. Furthermore, polarization measurements yield only a
two-dimensional projection of the spin axis; the true
three-dimensional orientation remains uncertain. Nevertheless, the
data are suggestive. Five of the six youngest and six of the eight
youngest pulsars have spin axes aligned with their movement. Only four
of the twelve older pulsars have spin axes aligned with their
movement. Similarly, none of the eight youngest pulsars are rotating
perpendicular to the galactic plane, but four of the twelve older
pulsars are in a perpendicular orientation. Perhaps even more
interesting, of the four older pulsars that are rotating in alignment
with their movement vector, two are also perpendicular to the galactic
plane, and the other two are relatively distant from the galactic
plane ($|z|$ values of 0.555 and 0.858~kpc, where the mean is 0.300
and the std dev is 0.235).

The \citet{Johnston2005} data supports two of the predictions made by
the mean field analysis---that spinning objects will have some
probability of becoming aligned vertically to the galactic plane, and
probability increases with proximity to the galactic plane.  However,
there are two significant shortcomings with the data. The first, as
already noted, is that the data size is small. The second is that it
does not explain the torque mechanism. Specifically, it does not speak
to how a force as weak as planar gravity might cause a pulsar's spin
axis to change, or why (e.g.,) the Sun's spin axis has changed so
little over a much longer timespan.

The most readily available energy source to power the evolution of a
pulsar's spin axis is the pulsar's own spin. This, combined with the
\citet{Johnston2005} data and the mean field analysis from the
previous section, makes a prediction that can be tested against the
larger \citet{Johnston2023} dataset, which includes timing data for
over 1,200 pulsars. Pulsars closer to the galactic plane are more
likely to be in a planar gravity field and/or are likely to be in more
planar gravity fields---if the pulsars' spin axes move in response to
planar gravity fields, and that movement is powered by spin, then
pulsars closer to the galactic plane should experience higher rates of
spin deceleration. The following subsections discuss the self-torquing
mechanism and test this hypothesis via the \citet{Johnston2005} data
set.


\subsection{The Dynamical Torque: Location-Dependent Spin-Down}

Shifting a pulsar's spin axis by tens of degrees over millions of
years is substantial work; there must be a proportionate energy source.
Planar gravity is weak; the planar
gravitational field cannot directly supply the energy needed to
reorient a rapidly spinning neutron star. Instead, the
energy comes from the pulsar's own rotation, with the planar field
serving only to define the preferred direction.

When a pulsar spins, it creates a coupling to any external
gravitational field that has a preferred direction. The multitude of
planar gravity fields near the the galactic plane, many of which will
be nearly parallel to the plane, provide precisely such a preferred
direction. When the pulsar's spin axis is misaligned to the fields that
cross it, the pulsar's frame-dragging interacts with the external
field to produce a torque. The torque draws power from the pulsar's spin.

The rotational kinetic energy of a spinning object is $E = \frac{1}{2}
I \omega^2$, where $I$ is the moment of inertia and $\omega$ is the
angular velocity. The coupling between a spinning object and an
external gravitational field scales with $\omega$ (faster rotation
means stronger frame-dragging), and the power dissipated---energy
extracted per unit time---scales with $\omega^2$. This is why
rapidly-spinning pulsars can significantly shift their spin axes over
millions of years: they have enormous rotational energy and a strong
coupling to the external field.

The Sun, by contrast, rotates with a period of approximately 25
days---roughly $10^8$ times slower than a typical pulsar. With $\omega$ so small,
the Sun's coupling to any intersecting planar fields is negligible, and its
available rotational energy is smaller by a factor of $10^{16}$. Despite
residing in the galactic disk for 4.6 billion years, the Sun has
experienced negligible gravitational torquing. Its modest $7^\circ$ axial
tilt relative to the ecliptic (itself inclined $60^\circ$ to the galactic
plane) reflects this weak coupling.

\subsection{Statistical Evidence: Spin-Down Rates}

The Johnston 2005 sample, while providing direct spin axis
measurements, is too small for us to draw robust statistical
conclusions regarding spin axis migration. However, we can leverage
the fact that pulsars are consuming rotational energy when migrating
their spin axes to generate a more-easily tested hypothesis.

Per the prediction made by our emergent behavior analysis and
evidenced by the Johnston 2005 data, there are more planar fields near
the galactic plane, both because there are more objects there but also
because the planar fields generated there are more likely to be
approximately parallel to the galactic plane. This both creates more
opportunities for pulsars to intersect planar gravity fields and
increases their dwell time within each field, since the fields are
oriented nearly-parallel to the pulsars' orbital planes. Since
torquing the spin axis expends spin energy, there should be a
measurable correlation between the normalized spin-down rate
$\dot{P}/P$ and distance from the plane $|z|$. This hypothesis can be
tested against a pulsar data set that does not include the axis of
rotation.

\citet{Johnston2023} published timing solutions for over 1,200 pulsars
observed with MeerKAT, including precise measurements of spin
frequency $F_0$ and its derivative $F_1$. From this set we excluded 22
pulsars that are distant from the galactic plane; we discuss them
separately below. In addition, we excluded the following types of
samples in accordance with pulsar analysis norms: millisecond pulsars
($P < 30$ ms) or with negative $\dot{P}$ (which have different
spin-down physics due to binary recycling or orbital contamination);
pulsars with $\tau < 10$ kyr (such very young pulsars have unsettled
spin-down behavior); and pulsars with missing $F_0$ and $F_1$
data. This left 1,191 pulsars to analyze.


\begin{figure}[t]
\includegraphics[scale=0.5]{figures/pdot_over_p_vs_z_paper.png}
\caption{Pulsar Spin Deceleration Rate vs. Distance from the
  Galactic Plane}
\note{Pulsars spin down more slowly the further they are from galactic
  plane, as predicted by the planar gravity model, due to there being
  fewer planar gravity fields further from the galaxy's orbital
  plane. $N$ is the number of pulsars in each bin.
}
\label{f:spindown_vs_z}
\end{figure}

We compute $\dot{P}/P = -F_1/F_0$ for each pulsar. Combined with
distance estimates from dispersion measures, we calculate each
pulsar's height $|z|$ above the galactic plane. Then we examined the
relationship between normalized spin-down and $|z|$.
Figure~\ref{f:spindown_vs_z} shows the result. The correlation is
striking: binning by $|z|$ reveals an order-of-magnitude variation in
median $\dot{P}/P$ for pulsars proximate to the galactic plane versus
those between 0.7 and 1.0~kpc from the galactic plane. The Spearman
rank correlation coefficient is $\rho = -0.403$ with $p = 7.59 \times
10^{-48}$.

Dense stellar environments amplify gravitational braking through two
mechanisms. First, the greater concentration of spinning objects near
the galactic plane means more planar gravity fields acting on any
individual pulsar. Second, and more subtly, these fields are never
perfectly aligned with one another---each object's spin axis differs
slightly from its neighbors'. The ``galactic planar gravity'' is not a
single coherent field but rather a population of
approximately-parallel fields. A pulsar near the plane therefore
cannot converge to a single aligned state; it is perpetually adjusting
to a shifting collection of conflicting fields. This prevents the torque
from ever vanishing entirely, maintaining elevated $\dot{P}/P$ even
for pulsars that have been near the plane for extended periods. The
effect is self-consistent: denser regions produce both stronger net
planar gravity \emph{and} more directional conflict, ensuring that pulsars
in these environments spin down faster and never fully
equilibrate. This may explain why the correlation between $\dot{P} /
P$ and $|z|$ remains strong across the entire disk population rather
than showing a saturated ``aligned'' subpopulation at low $|z|$.

While the data appears to support our hypothesis, spin-down rate is
also dependent upon the pulsar's intrinsic properties, and
newly-formed pulsars are more likely to be found near the galactic
plane simply because that is where massive stars—the progenitors of
pulsars—reside. Thus, Fig.~\ref{f:spindown_vs_z} could merely
illustrate that $|z|$ and pulsar age are correlated: young pulsars
have not yet migrated far from their birthplaces, and young pulsars
spin down faster because they are young. To address this concern, we
examine a quantity that should be independent of location under
standard physics: the pulsar's magnetic field strength.

Under standard magnetic dipole braking, the relationship between
spin-down rate, period, and magnetic field is:
\begin{equation}
\dot{P} = \frac{8\pi^2 B^2 R^6}{3c^3 I P}
\end{equation}
where $B$ is the surface magnetic field strength, $R$ is the neutron
star radius, $c$ is the speed of light, and $I$ is the moment of
inertia. Rearranging to solve for $B$:
\begin{equation}
B = \sqrt{\frac{3c^3 I P \dot{P}}{8\pi^2 R^6}}
\label{e:B_field}
\end{equation}

The magnetic field strength is set at birth during core collapse and
should be independent of where the pulsar subsequently migrates. Birth
location (near the galactic plane where massive stars reside) does not
determine final location after millions of years of migration with
kick velocities of 100--400 km/s. Therefore, $B$ calculated from
Equation~\ref{e:B_field} should show no correlation with current
$|z|$.


\begin{figure}[t]
\includegraphics[scale=0.4]{figures/B_field_vs_z_paper.png}
\caption{Magnetic Field Strength vs. Distance from the Galactic Plane}
\note{The magnetic field, inferred from standard dipole braking
  formula, shows 2.5$\times$ systematic variation with $|z|$. The
  right panel shows median, $75^\text{th}$ and $25^\text{th}$
  values. Given that magnetic field is set at birth and is independent
  of location, this confirms location-dependent braking
  ($\rho = -0.29$, $p < 5 \times 10^{-25}$).}
\label{f:B_field_vs_z}
\end{figure}


Figure~\ref{f:B_field_vs_z} shows the relationship between inferred
magnetic field strength and distance from the galactic plane for 1,201
MW disk pulsars. The left panel reveals a strong anticorrelation:
pulsars near the plane have systematically higher inferred $B$-fields
than those far from the plane (Spearman $\rho = -0.29$, $p < 5 \times
10^{-25}$). The right panel shows median $B$-fields by distance bin,
revealing a 2.5-fold variation between near-plane ($|z| < 0.1$~kpc)
and far-from-plane ($|z| > 0.7$~kpc) pulsars.

This correlation cannot be explained by age-location effects. If
younger pulsars remain near the plane while older pulsars migrate to
higher $|z|$, we would expect no systematic variation in $B$ (which is
age-independent) but would observe higher $\dot{P}/P$ at low $|z|$ due
to the age distribution alone. Instead, we observe that $B$ itself
varies with location, indicating that pulsars near the plane
experience additional braking beyond what standard magnetic dipole
radiation predicts. When we calculate $B$ using the standard formula,
this extra braking appears as an artificially elevated magnetic field.

The $B$-field variation isolates the age-independent component of
location-dependent braking. Since $\dot{P}/P \propto B^2/P^2$ under
standard dipole braking, the 2.5-fold $B$ variation should produce a
6.2-fold $\dot{P}/P$ variation. In logarithmic terms (appropriate for 
multiplicative effects), this accounts for approximately 62\%
of the observed 19-fold $\dot{P}/P$ variation with $|z|$. The
remaining 38\% could reflect age-location correlation,
period-dependent effects, or measurement uncertainty, but the
$B$-field signature alone provides strong evidence for
location-dependent braking.

\begin{table}[h]
\centering
\caption{Correlation between magnetic field strength and distance from 
galactic plane within period bins, demonstrating the location-dependent 
braking effect is independent of pulsar rotation rate.}
\label{t:B_field_period_control}
\begin{tabular}{lccc}
\hline
Period Range (ms) & $N$ & Spearman $\rho$ & $p$-value \\
\hline
31--361   & 396 & $-0.425$ & $8.7 \times 10^{-19}$ \\
361--745  & 408 & $-0.389$ & $3.5 \times 10^{-16}$ \\
745--8510 & 396 & $-0.291$ & $3.7 \times 10^{-9}$ \\
\hline
\multicolumn{2}{l}{Mean $\rho$ across bins:} & $-0.368$ & \\
\multicolumn{2}{l}{Overall $\rho$ (uncontrolled):} & $-0.292$ & $5.0 \times 10^{-25}$ \\
\hline
\end{tabular}
\note{The correlation between inferred $B$-field and $|z|$ persists within 
each period bin, demonstrating the effect is not driven by period-dependent 
artifacts. The mean within-bin correlation ($\rho = -0.37$) is stronger than 
the overall correlation ($\rho = -0.29$), indicating period adds noise rather 
than confounding the location-dependent signal.}
\end{table}

To verify this result is not confounded by period-dependent effects,
we examined the $B$ versus $|z|$ correlation within period bins. If
the correlation arose from some period-dependent artifact, it should
weaken or disappear when controlling for $P$. Instead, the correlation
strengthens: within each of three period bins spanning 31--8510 ms,
the Spearman correlation between $\log(B)$ and $|z|$ ranges from $\rho
= -0.29$ to $\rho = -0.43$ (mean $\rho = -0.37$), demonstrating the
effect is robust across the pulsar population regardless of rotation
rate. 

One might ask whether the location-dependent spin-down reflects some
non-gravitational effect correlated with proximity to the
plane—perhaps interactions with the denser interstellar
medium. Section~\ref{s:mw_vertical} presents independent evidence that
argues against such alternatives.


\subsection{Control Groups: Milky Way Escapees and Magellanic Cloud Pulsars}

The Johnston 2023 sample includes 22 pulsars with $|z| > 1$~kpc. We
excluded them from the above analysis as outliers, but they offer
interesting insights into the alignment phenomenon.  Examining them
reveals that they fall into two categories: they are either in the
Large and Small Magellanic Clouds (they are not Milky Way pulsars at
all), while the others are ``escapees'' from the Milky way (pulsars
existing in open space, outside the confines of the galaxy).

Separating these populations provides a powerful test:

\begin{itemize}
\item \emph{Milky Way escapees} (high $|z|$, not toward Magellanic
  Clouds): These pulsars have presumably been ejected from the
  galactic disk by supernova kicks and now reside outside the Milky
  Way's (or any other galaxy's) disk. They show very low $\dot{P}/P$---minimal
  braking in a planar-gravity-sparse environment.

\item \emph{Magellanic Cloud pulsars} (high $|z|$ from MW plane, but
  within LMC/SMC): Despite being far from the Milky Way's plane, these
  pulsars reside within \emph{another} disk galaxy's planar
  field. They show high $\dot{P}/P$, comparable to Milky Way disk
  pulsars.
\end{itemize}

This pattern is precisely what the gravitational braking hypothesis
predicts: spin-down depends not on distance from the Milky Way
specifically, but on proximity to \emph{any} disk galaxy's orbital
plane, with the associated density of planar gravity fields.
The Magellanic Cloud pulsars experience braking
from their host galaxies' planar fields; the true Milky Way escapees,
isolated in intergalactic space, experience almost none.

\subsection{A Note on ``Characteristic Age''}

Standard pulsar astronomy interprets $\tau = P/(2\dot{P})$ as a proxy
for age, assuming spin-down is dominated by location-independent
magnetic dipole radiation. Under this interpretation, pulsars with
high $\tau$ are ``old'' and those with low $\tau$ are ``young.''

Our results challenge this interpretation. If gravitational braking
contributes significantly to spin-down, and if braking strength
depends on location, then $\tau$ conflates two effects: time since
birth and cumulative exposure to planar fields. A pulsar
ejected from the disk shortly after birth might retain high spin (low
$\tau$) not because it is young, but because it has spent most of its
life in a low-braking environment.

The Milky Way escapees in our sample may be cases in point.
J1632$-$1013, with $\tau = 172$ Myr and $|z| = 1.25$~kpc, may not be
172 million years old. Having escaped the disk quickly, it is
experiencing minimal braking---the small $\dot{P}$ relative to its $P$
makes the pulsar appear older than it is. 172 million years is
remarkably old for a pulsar; most die within 50 Myr, and the oldest
pulsars with independently verified ages (from supernova remnant
associations) are typically 10-20 Myr. Without independent age
constraints (e.g., supernova remnant associations), characteristic age
should be interpreted cautiously for pulsars outside galactic disks.

\subsection{Conclusions from Pulsar Analysis}

The pulsar data provide strong, independent support for the emergent
behavior resulting from planar gravity in a rotating galaxy hypothesis:

\begin{enumerate}
\item \emph{Spin-down rate is location-dependent.} Pulsars near the
  galactic plane spin down up to 10 times faster than those far from
  the plane, with overwhelming statistical significance ($p <
  10^{-43}$).

\item \emph{Perpendicular alignment is observed.} Among the small
  sample of pulsars with measured spin axes, there is a suggestive
  excess of perpendicular orientations, which appears to correlate
  with both age (very young pulsars are not aligned) and $|z|$
  (pulsars close to the galactic plane are more likely to be aligned).

\item \emph{The effect is universal to disk galaxies.} Magellanic
  Cloud pulsars, despite their distance from the Milky Way plane, show
  high spin-down rates consistent with braking in their host galaxies'
  planar fields. True intergalactic pulsars show minimal braking.

\item \emph{The energy source is the stellar object's own spin.} The
  $\omega^2$ dependence of gravitational torque explains why pulsars
  (fast rotators) align while the Sun (slow rotator) does not, and why
  spin-down accompanies alignment.
\end{enumerate}

These findings anticipate and complement the rotation curve
analysis of the following section. Where the rotation curves
demonstrate that planar gravity produces the flat acceleration curves
at galactic scales, the pulsar data demonstrate that the same planar
gravity results in measurable torques on individual spinning
objects. Two independent lines of evidence---galactic dynamics and
pulsar spin evolution---point to the same underlying phenomenon.


%==============================================================================
\section{The Gravitational Signature: Flat Rotation Curves}\label{s:sparc_validation}
%==============================================================================

The previous section established observational evidence that planar
gravity exists and operates dynamically: pulsar spin axes align with
the galactic plane, and spin-down rates vary with location in a manner
consistent with gravitational torquing. In this section, we turn to
the primary gravitational signature---the anomalously high rotational
velocities in disk galaxies. (\emph{NB}: in it is customary to refer
to ortibal speeds in rotating galaxies as \emph{rotational
velocities}---this should not be confused with pulsar spin rates,
discussed in the previous section.)

Our test bed is the SPARC (Spitzer Photometry and Accurate Rotation
Curves) database, which provides detailed mass distributions and
rotation curves for 175 disk galaxies spanning four orders of
magnitude in luminosity and surface brightness. This dataset underpins
McGaugh et al.'s \citep{McGaugh2016} Radial Acceleration Relation
(RAR), an empirical function relating observed centripetal
acceleration to that predicted from baryonic mass alone. The RAR
provides a quantitative benchmark: any successful explanation for flat
rotation curves must reproduce its predictions. We will show that
planar gravity meets this test, and moreover explains why the RAR
takes the functional form it does---the relation emerges naturally from
the geometry of aligned spinning masses.

A significant challenge confronts any attempt to predict planar
gravity from first principles. The strength of the collective planar
field depends on numerous factors: the distribution of stellar masses;
the fraction of dark compact objects (neutron stars, black holes)
contributing mass but not light; the spin rates of individual
objects; the degree to which spin axes have achieved alignment; and the
precise relationship between an object's angular momentum and its
contribution to the planar field. None of these quantities are
directly observable for external galaxies.

We overcome this challenge by recognizing that flat, rotating disk
galaxies represent systems that followed similar evolutionary paths
and have reached comparable dynamical states. Rather than modeling
individual stellar populations, we parameterize the aggregate
relationship between a galaxy's baryonic mass and its planar gravity
contribution using two population-level parameters: $\varepsilon$,
representing the base efficiency with which baryonic mass generates
planar gravity; and $\alpha$, capturing how this efficiency varies
with galactocentric radius. This approach absorbs our ignorance about
the underlying distributions into parameters that can be constrained
empirically.

The following subsections develop this framework. We first define the
parameters $\varepsilon$ and $\alpha$ and their physical
interpretation. We then describe the SPARC dataset and our fitting
methodology. Finally, we present results both in
aggregate---demonstrating that planar gravity reproduces the RAR---and
for individual galaxies, examining cases where the model performs best
and where discrepancies point to interesting physics.


\subsection{The Population Parameters $\varepsilon$ and $\alpha$}

To predict rotation curves from planar gravity, we must calculate the
cumulative gravitational field from all spinning masses in a
galaxy. In principle, this requires summing contributions from
billions of individual objects---stars, neutron stars, black
holes---each with its own mass, spin rate, and axis orientation. In
practice, we can treat the galaxy as a continuous mass distribution
and parameterize our ignorance about the underlying stellar
populations.

\paragraph{The continuum approximation} A typical disk galaxy
contains $10^{10}$--$10^{11}$ spinning masses distributed in a thin,
approximately axisymmetric disk. At this scale, individual
contributions blur into a smooth field. We therefore model the galaxy
as a two-dimensional surface mass density $\Sigma(R)$, where $R$ is
the galactocentric radius. This is the same approximation used in
standard rotation curve analysis; we add a planar gravity
component to the usual Newtonian calculation.

\paragraph{The coupling parameter $\varepsilon$} Consider a mass
element $\mathrm{d}m$ at radius $R'$ in the disk. Its
Newtonian contribution to gravity falls off as $1/r^2$. Its planar
gravity contribution, arising from the mass element's spin, falls off
as $1/r$. We parameterize the strength of this planar contribution as:

\begin{equation}
\mathrm{d}g_{\text{planar}} = \varepsilon \cdot G \cdot \frac{\mathrm{d}m}{r}
\end{equation}
where $r$ is the distance from the mass element to the test point and $G$
is Newton's constant.

The dimensionless parameter $\varepsilon$ absorbs the parameters we
cannot measure directly: the distribution of spin rates across the
galaxy's spinning masses, the fraction of spin axes aligned
perpendicular to the disk, the relative contributions of luminous and
dark objects (stars versus neutron stars and black holes), and the
fundamental relationship between angular momentum and planar field
strength. Rather than modeling each factor independently, we treat
$\varepsilon$ as an empirical parameter to be determined from rotation
curve fits.

\paragraph{The radial variation parameter $\alpha$}
We do not expect $\varepsilon$ to be constant across a galaxy. Spin
alignment is a dynamic process driven by gravitational torques from
the collective planar field. The efficiency of this alignment depends
on local conditions. In the inner galaxy, the disk is dense and
dynamically hot. Frequent gravitational interactions perturb spin
axes, working against alignment. The disk also has greater vertical
thickness, meaning spin axes point in more varied directions even
among ``aligned'' objects.  In the outer galaxy, conditions favor
alignment. The disk is thinner, so aligned objects contribute more
coherently to the planar field. The disk is also sparser, reducing the
gravitational noise from close encounters that would otherwise perturb
spin axes. Both effects increase the effective planar gravity
generated per unit mass at larger radii.

We model this radial dependence as:
\begin{equation}
\varepsilon_{\text{eff}}(R') = \varepsilon \cdot \exp\left(\alpha \frac{R'}{R_d}\right)
\label{e:epsilon_eff}
\end{equation}
where $R_d$ is the disk scale length and $\alpha$ characterizes the
rate at which alignment efficiency increases with radius. A positive
$\alpha$ indicates improving alignment toward the outer disk, as
expected from the physical arguments above.

Crucially, we do not fit $\varepsilon$ and $\alpha$ individually to
each galaxy. Doing so would grant the model excessive flexibility,
potentially allowing it to match observations through parameter tuning
rather than physical accuracy. Instead, we determine these parameters
from a universal meta-function that takes a single observable input:
the galaxy's disk mass. This approach yields zero free parameters per
galaxy, making planar gravity a calculable function over observable
properties, akin to equations of state or scaling laws rather than
empirical curve-fitting. The meta-function itself is derived
empirically from the SPARC sample, but once established, it applies
uniformly to all disk galaxies. The physical motivation is
straightforward: given that rotating galaxies obey common structural
constraints, their disk mass correlates with stellar density and
dynamical history, which in turn determine the degree of spin
alignment achieved.

In summary, the variables $\varepsilon$ and $\alpha$ characterize a
galaxy's planar gravity field. The base parameter $\varepsilon$ sets
the overall coupling strength between baryonic mass and planar
gravity; it is primarily determined by the galaxy's population of
spinning masses and dynamical history. The radial parameter $\alpha$
modulates the spatial distribution of planar gravity, capturing the
systematic improvement in alignment efficiency toward the outer
disk. In the following subsections, we describe how these parameters
are constrained by fitting observed rotation curves.

\subsection{Analysis Methodology}

Our analysis proceeds in two phases. First, we fit $\varepsilon$ and
$\alpha$ individually to each galaxy in the SPARC sample, minimizing
the discrepancy between predicted and observed rotation curves. This
phase reveals how the optimal parameters vary across the galaxy
population. Second, we search for systematic relationships between
these fitted parameters and observable galaxy properties. If such
relationships exist, they can be encoded in a meta-function that
predicts $\varepsilon$ and $\alpha$ from observables alone, yielding a
model with zero free parameters per galaxy.

\paragraph{Baryonic mass distribution}
For baryonic gravitational acceleration, we adopt the same approach as
\citet{McGaugh2016}. The SPARC database \citep{Lelli2016} provides
rotation curves and mass models for 175 disk galaxies observed with
Spitzer at 3.6~$\mu$m. For each galaxy, the database includes
pre-computed circular velocity contributions from three baryonic
components: $V_{\text{gas}}(R)$ from neutral hydrogen observations,
$V_{\text{disk}}(R)$ from the stellar disk, and $V_{\text{bul}}(R)$
from the bulge where present. The total baryonic velocity and
acceleration follow from quadrature addition:
\begin{equation}
V_{\text{bar}}^2 = V_{\text{gas}}^2 + V_{\text{disk}}^2 + V_{\text{bulge}}^2
\end{equation}
and $g_{\text{bar}} = V_{\text{bar}}^2 / R$. This quantity represents
the centripetal acceleration that would be observed if Newtonian
gravity were the only gravitational source.

\paragraph{Planar gravity calculation}
To calculate the planar gravitational acceleration $g_{\text{planar}}$
at radius $r$, we must sum the $1/r$ contributions from all mass
elements in the disk. For a test particle orbiting at radius $r$, a
mass element $\mathrm{d}m$ located at radius $R'$ and azimuthal angle
$\phi'$ contributes planar gravity inversely proportional to the
separation distance. Integrating over the full disk yields:
\begin{equation}
g_{\text{planar}}(r) = \int_0^{R_{\text{max}}} \int_0^{2\pi}
\frac{\varepsilon_{\text{eff}}(R') \cdot G \cdot \Sigma(R') \cdot R'
  \, dR' \, d\phi'}{\sqrt{r^2 + R'^2 - 2rR' \cos \phi'}}
\label{e:planar_integral}
\end{equation}
where $\Sigma(R')$ is the surface mass density at radius $R'$, derived
from the SPARC velocity decomposition, and the denominator is the
distance from the mass element to the test point. The integration
extends to $R_{\text{max}}$, chosen to be several times the outermost
observed data point. (We assume that spinning objects beyond this
point are both so sparse and so distant as to not affect results.)
This integral is evaluated numerically for each radius at which
rotation velocity is measured.

The predicted rotation velocity combines Newtonian and planar contributions:
\begin{equation}
V_{\text{pred}}^2 = V_{\text{bar}}^2 + r \cdot g_{\text{planar}}
\end{equation}
which can be compared directly to observed velocities $V_{\text{obs}}$.

\paragraph{Parameter fitting and meta-function discovery}
For each galaxy, we search for the values of $\varepsilon$ and
$\alpha$ that minimize the root-mean-square (RMS) residual between
$\log_{10}(V_{\text{pred}})$ and $\log_{10}(V_{\text{obs}})$ across
all measured radii. We employ gradient descent optimization,
initializing from multiple starting points to avoid local minima.

With optimal parameters determined for each galaxy, we examine their
distribution across the sample. The key finding is that $\varepsilon$
correlates strongly with disk mass, following a log-linear
relationship:
\begin{equation}
\log_{10}(\varepsilon) = -20.32 - 0.669 \cdot \log_{10}(M_{\text{disk}})
\label{e:metafunction}
\end{equation}
This inverse correlation has a natural physical interpretation: more
massive disks have higher stellar densities and encounter rates, which
disrupt spin alignment and reduce planar gravity efficiency. The
parameter $\alpha$ shows no significant correlation with disk mass or
other observable properties.  The emergent behavior analysis predicted
that the alignment fraction increases as the galactic disk
thins. While the optimal value of $\alpha$ varies for each galaxy and
does not correlate with any of the galaxy attributes available in the
SPARC database, the performance of the planar gravity model was
improved for all galaxies using the median $\alpha = 0.20$.

With this meta-function established, the model becomes fully
predictive: given only a galaxy's disk mass, we compute $\varepsilon$
from Equation~\ref{e:metafunction}, set $\alpha = 0.20$, and
calculate the rotation curve with no adjustable parameters.

\subsection{Rotation Curve Prediction Results}

Applying the meta-function (Equation~\ref{e:metafunction}) to predict
$\varepsilon$ from disk mass alone, with $\alpha = 0.20$ held
constant, we calculate rotation curves for 172 galaxies. (Three of the
175 galaxies in the SPARC database are truly edge-on, and so no
velocity information is available for them.)  Table~\ref{t:summary}
summarizes the results.

\begin{table}[t]
\caption{Model performance summary}
\begin{tabular*}{\textwidth}[]{p{6cm}@{\extracolsep\fill}c}
\toprule
Statistic & Value \\
\midrule
Median RMS & 0.108 dex \\
Mean RMS & 0.124 dex \\
Standard deviation & 0.075 dex \\
Galaxies with RMS $< 0.2$ dex & 136/158 (86\%) \\
\bottomrule
\end{tabular*}
\note{Performance of the planar gravity model across 172 of the 175
  galaxies in the SPARC database. An RMS of 0.108~dex corresponds to velocity predictions
  accurate to within $\sim$28\% across all radii, with no free
  parameters adjusted per galaxy.}
\label{t:summary}
\end{table}

The median RMS residual of 0.108~dex is comparable to the 0.13~dex
scatter reported by \citet{McGaugh2016} for the RAR, and better than
typical dark matter halo fits which require multiple free parameters
per galaxy. This level of agreement---achieved with zero free
parameters per galaxy---approaches the scatter attributed to
measurement uncertainty alone. The 18\% of galaxies with RMS $> 0.2$
dex are not random failures: they cluster among systems with known
dynamical complications (strong bars, tidal interactions, AGN
activity).

\paragraph{Galaxy selection} Our exclusion criteria differ from
\citet{McGaugh2016} in a physically motivated way. McGaugh excluded 22
galaxies (reducing from 175 to 153), rejecting face-on systems ($i <
30^\circ$) due to uncertain $\sin(i)$ corrections, plus galaxies with
asymmetric rotation curves. We employ a two-tier system: stricter
exclusions for meta-function calibration (where clean measurements are
essential), and minimal exclusions for model evaluation (where we test
predictions across the widest possible range).

For calibration, we exclude 17 galaxies in seven categories:
edge-on/high-inclination systems (4 galaxies), where the rotation
curve measures line-of-sight velocity through the disk rather than
in-plane circular motion; strongly barred galaxies (2), where
bar-dominated kinematics produce non-circular orbits; AGN hosts (2),
where non-gravitational energy sources affect central kinematics;
interacting pairs (2), where tidal forces randomize spin axes;
disturbed morphologies (3); chaotic irregulars lacking coherent disk
structure (2); and data quality issues (2). We imposed more
restrictions on the galaxies we calibrated over (reducing the galaxy
set from 175 to 158) than we imposed on the evaluation set (172
galaxies) to avoid having outlier cases pollute the model.

\subsection{Comparison with the Radial Acceleration Relation}

Direct comparison with \citet{McGaugh2016} requires methodological
alignment. We therefore apply the RAR to the same 172 galaxies, using
the RAR to generate velocity predictions for each data point in the
SPARC database.

\begin{table}[t]
\caption{Comparison with RAR}
\begin{tabular*}{\textwidth}[]{p{5cm}@{\extracolsep\fill}cc}
\toprule
Metric & RAR & Planar Gravity \\
\midrule
Median RMS & 0.110 dex & 0.108 dex \\
Mean RMS & 0.134 dex & 0.124 dex \\
Standard deviation & 0.088 dex & 0.069 dex \\
Better fit (galaxy count) & 64 & 108 \\
\bottomrule
\end{tabular*}
\note{Planar gravity generates slightly more accurate predictions of
  velocity than the RAR, but both are within the measurement
  uncertainty of the SPARC data.}
\label{t:comparison}
\end{table}

RAR is a purely empirical relation, while planar gravity derives from
a physical model whose single meta-function was calibrated on this
dataset. That a physical model is an improvement over the empirical
relation suggests it captures the underlying physics.

\paragraph{Out-of-sample performance} Both models generalize beyond
their training data, but planar gravity demonstrates notably stronger
out-of-sample performance. To assess this, we evaluated each model on
galaxies excluded from its calibration set, using the other model
(trained on those systems) as a benchmark.

\begin{table}[h]
\centering
\caption{Out-of-sample performance comparison}
\label{t:out_of_sample}
\begin{tabular}{lcc}
\hline
Test Set & Planar Gravity & RAR \\
\hline
Planar-excluded (14 galaxies) & 0.132 dex & 0.172 dex \\
RAR-excluded (22 galaxies) & 0.177 dex & 0.185 dex \\
\hline
\end{tabular}
\note{Median RMS residuals on galaxies excluded from each model's
  calibration. Planar gravity outperforms on both test sets.}
\end{table}

On the 14 galaxies excluded from planar calibration (systems with
strong bars, AGN activity, tidal interactions, or chaotic kinematics),
the meta-function wins 85.7\% of pairwise comparisons. On the 22
galaxies excluded from RAR's calibration (face-on systems and
low-quality curves), planar gravity wins 59.1\%. The planar model also
exhibits lower overall variance ($\sigma = 0.075$ dex versus 0.088
dex), fewer catastrophic failures (22 versus 25 galaxies with RMS $>
0.20$ dex), and more excellent fits (13 versus 11 galaxies with RMS $<
0.05$ dex). These results suggest the mass-dependent meta-function
captures physical relationships that generalize to morphologically
diverse systems.


\begin{figure}[t]
\includegraphics[scale=0.4]{figures/IC2574_analysis.png}
\caption{Planar and RAR Predictions for IC2574}

\note{IC2574 is a gas-rich late-type spiral (Sm) galaxy with rotation
  curve data extending well beyond its stellar disk.}

\label{f:IC2574}
\end{figure}

Figure~\ref{f:IC2574} shows the performance of the planar and RAR
models on IC2574, a gas-rich late-type spiral (Sm) galaxy whose
$M_\text{disk}$ is $0.051 \times 10^{10}\ \text{M}_\odot$, which puts
it at $\approx 25^\text{th}$ percentile in the SPARC database. Its
$R_\mathrm{d}$ is at 2.78~kpc; 91\% of a galaxy's disk-mass is
enclosed at around $4 \times R_\mathrm{d}$, so there is still
measurable disk presence at the upper end of the SPARC data ($\approx
10$~kpc). The planar model achieves 0.042 dex RMS, outperforming RAR's
0.200 dex.


\begin{figure}[t]
\includegraphics[scale=0.4]{figures/ESO116-G012_analysis.png}
\caption{Planar and RAR Predictions for ESO116-G012}

\note{ESO116-G012 is a mid-mass (~$50^\text{th}$ percentile),
  stellar-dominated late-type spiral (Sd) galaxy. $R_\mathrm{d}$ is
  only 1.51~kpc (less than the much-smaller IC2574).}

\label{f:ESO116_G012}
\end{figure}

Figure~\ref{f:ESO116_G012} shows ESO116-G012, a stellar-dominated
late-type spiral (Sd) at the $50^\text{th}$ percentile in disk mass
($M_\text{disk} = 0.215 \times 10^{10}\ \text{M}_\odot$). This galaxy
reveals an important limitation: the planar model outperforms RAR in
the inner disk (up to $\sim$4~kpc), but RAR is superior in the outer
regions where gas dominates. At 4~kpc, gas contributes approximately
11\% of the local mass; by 7~kpc, this rises to over 26\%.

This pattern reflects a key methodological difference: the planar
meta-function uses disk mass alone to predict $\varepsilon$, while RAR
uses total baryonic gravity (bulge, disk, and gas combined). In
gas-rich outer regions, this gives RAR an advantage. Incorporating gas
mass into the meta-function could improve outer-region predictions---a
direction for future work.

%==============================================================================
% \section{Discussion}\label{s:discussion}
%==============================================================================

\subsection{Summary}

The planar gravity model, parameterized by a simple mass-dependent
function, reproduces disk galaxy rotation curves with accuracy
comparable to the Radial Acceleration Relation: median RMS residual
0.108~dex, zero free parameters per galaxy, 86\% of galaxies fit within
0.2~dex.

Unlike RAR, which is purely empirical, the planar gravity model
derives from a physical mechanism: aligned spinning masses generating
a $1/r$ gravitational component. The inverse relationship between
$\varepsilon$ and disk mass has a plausible physical interpretation:
more massive galaxies have higher encounter rates and more dynamical
processing, disrupting spin alignment and reducing planar gravity
efficiency.

%==============================================================================
\section{Discriminating Test: Milky Way Vertical Velocity Gradient}\label{s:mw_vertical}
%==============================================================================

The preceding sections established evidence for planar gravity through
two approaches: pulsar observations demonstrating the mechanism at the
individual-object level (Sec.~\ref{s:pulsars}), and SPARC rotation
curves demonstrating the collective effect at galactic scales
(Sec.~\ref{s:sparc_validation}). We now present a third, independent line of
evidence: a discriminating test where planar gravity and dark matter
make opposing predictions.  The test is straightforward. If flat
rotation curves arise from a spherical dark matter halo, then circular
velocity at galactocentric radius $R$ should be independent of height
$|z|$ above the disk—the halo provides equal gravitational
acceleration at all points equidistant from the galactic center. If
instead flat rotation curves arise from planar gravity confined to the
disk, then stars at greater $|z|$ lie partially or wholly outside
the planar field and should orbit more slowly at the same $R$.

Both the mean-field analysis (Sec.~\ref{s:emergent}) and the pulsar
observations make specific predictions about this vertical
structure. The emergent behavior framework predicts that planar
gravity should be strongest near the galactic midplane, where the
density of aligned spinners is highest and the coupling parameter
$\alpha$ dominates the disturbance parameter $\beta$. The pulsar
spin-down data (Fig.~\ref{f:spindown_vs_z}) provides empirical
confirmation: $\dot{P}/P$ decreases by approximately an order of
magnitude between $|z| \approx 0$ and $|z| \approx 0.7$~kpc,
indicating rapid attenuation of the planar field with height. If this
attenuation reflects the true vertical profile of planar gravity,
circular velocities should show a corresponding decrease with $|z|$.

\citet{Wang2023} mapped Milky Way stellar kinematics using Gaia DR3
data extending to $R \approx 30$~kpc. Their analysis reveals precisely
the pattern planar gravity predicts. For $R < 15$~kpc, azimuthal
velocity $V_\phi$ shows marked dependence on height: stars in the
midplane ($|z| \approx 0$) orbit at $\sim$220--230~km/s, while stars
at $|z| = 2.5$~kpc orbit at $\sim$200--210km/s, and those at
$|z| = 4.5$~kpc orbit at only $\sim$150--180~km/s. The
authors note that this vertical gradient ``has, to our knowledge, not
been noticed before.''  The gradient varies with radius. It is modest
in the inner galaxy ($R \lesssim 10$~kpc), strengthens through the
outer thin disk ($R \sim 12-14$~kpc), and disappears beyond $R >
15$~kpc where curves at different heights converge. This structure is
consistent with planar gravity: the inner galaxy contains bulge
populations and thick disk that disrupt alignment, while the
12--14~kpc range corresponds to the outer thin disk where spin
alignment should be cleanest. The convergence beyond 15~kpc---where
the stellar disk becomes sparse, extended gas contributes Newtonian
but not planar gravity, and halo populations contaminate
samples---does not contradict the model but reflects the increasing
complexity of Milky Way structure at large radii.

Crucially, the MW velocity gradient provides independent confirmation
that the pulsar spin-down correlation (Sec.~\ref{s:pulsars}) reflects
planar gravity rather than some other location-dependent braking
mechanism. The pulsar analysis alone cannot exclude the possibility
that an unknown effect—perhaps related to the denser interstellar
medium near the plane—causes location-dependent spin-down unrelated to
gravitational torquing. But such an alternative would not predict a
corresponding gradient in stellar orbital velocities. The fact that
both pulsars (spinning down faster near the plane) and disk stars
(orbiting faster near the plane) show the same vertical dependence, in
the manner predicted by planar gravity, argues strongly that a single
mechanism explains both.

\section{Limitations}\label{s:limitations}

Two limitations warrant acknowledgment. First, we have not derived
planar gravity from first principles---it remains phenomenological.
While its properties (generated by spin, producing torque and shear
forces) strongly suggest it is a component of frame-dragging, we lack
a theoretical derivation from general relativity. The $1/r$ falloff,
however, is physically plausible. For a gravitational component
confined to a two-dimensional surface, flux conservation naturally
yields $1/r$ rather than Newtonian gravity's $1/r^2$. Alternatively,
the scaling may emerge from collective effects in disk
geometry. Regardless of microscopic origin, $1/r$ is precisely the
scaling required to produce constant rotation velocities at large
radii, suggesting the phenomenology captures real physics even absent
rigorous derivation.

Second, the model addresses only rotating disk galaxies. Phenomena at
larger scales---galaxy clusters and cosmological observations---remain
outside this work's scope and may require additional physics. That
said, one apparent challenge deserves mention. The Bullet Cluster
collision shows gravitational lensing mass separated from X-ray
emitting gas, which is widely interpreted as collisionless dark matter
particles passing through while gas collides
electromagnetically. However, stellar objects are also largely
collisionless in such encounters. Under planar gravity, the
gravitational signal follows the stellar populations because that is
where the aligned spinning masses reside. The observed separation of
lensing signal from gas is therefore consistent with planar gravity
without requiring particle dark matter. A full treatment of
cluster-scale dynamics lies beyond this paper's scope.

\section{Conclusion}\label{s:conclusion}

We have presented three independent lines of evidence that the flat
rotation curves observed in rotating galaxies arises from a geometric
property of gravitational fields: coherently aligned spin axes
producing a collective planar gravitational component with $1/r$
falloff.

First, pulsar observations demonstrate the mechanism's operation at the
individual-object level. Pulsars near the galactic plane spin down up to
ten times faster than those at high $|z|$ ($\rho = -0.38$, $p < 10^{-43}$),
with this location-dependent braking unexplainable by standard magnetic
dipole radiation. The small sample of pulsars with measured spin axes shows
suggestive evidence for progressive alignment toward perpendicularity with
age and proximity to the plane. Magellanic Cloud pulsars experience high
spin-down despite distance from the Milky Way plane, confirming that the
effect depends on any disk galaxy's planar field rather than Milky Way
proximity specifically.

Second, the model reproduces 172 SPARC galaxy rotation curves with median
RMS residual of 0.108~dex---matching the empirical Radial Acceleration
Relation's precision while deriving from physical principles rather than
curve-fitting. This performance is achieved with zero free parameters per
galaxy, expressing the additional acceleration as a deterministic function
of observable baryonic properties. The model naturally explains why dark
matter effects appear in rotating disks but not pressure-supported systems:
morphology, not mass, determines whether conditions allow spin alignment.

Third, Gaia DR3 data provide a discriminating test: at fixed
galactocentric radius, stars farther from the Milky Way's midplane
orbit more slowly than those near it. Spherical dark matter halos
predict no such vertical gradient; planar gravity confined to the disk
predicts exactly this pattern.

Planar gravity differs from particle dark matter in requiring no free
parameters per galaxy: the additional acceleration is a deterministic
function of observable baryonic properties. It differs from MOND in
deriving from a physical mechanism rather than phenomenological
modification of force laws. This mechanistic basis yields predictions
MOND cannot make---location-dependent pulsar spin-down, vertical
velocity gradients in disk galaxies---both of which are observed.

The success of the three predictions---individual pulsar dynamics,
collective galactic rotation curves, and Milky Way stellar rotation
speeds---from the same underlying mechanism provides strong evidence
that the planar gravity model captures real physics. Future work
should pursue theoretical derivation of the planar component from
general relativity, expanded pulsar spin axis surveys to strengthen
the alignment evidence, and investigation of whether related
mechanisms operate at larger scales beyond individual galaxies.

%==============================================================================
\bibliography{\bib}
%==============================================================================

\end{document}
