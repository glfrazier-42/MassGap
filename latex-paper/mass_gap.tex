\documentclass[letterpaper,11pt,leqno]{article}
\usepackage{paper}
\bibliographystyle{mass_gap}
\usepackage{dcolumn}
\usepackage{booktabs}
\newcolumntype{d}[1]{D{.}{.}{#1}}

% Paper title for PDF metadata:
\hypersetup{pdftitle={Title To Be Determined}}

% Path to BibTeX file:
\newcommand{\bib}{mass_gap.bib}

\begin{document}

\title{Title To Be Determined}

\author{Gregory L. Frazier \thanks{Independent researcher. The author
    thanks Claude (Anthropic) for assistance with analysis and
    manuscript preparation.}}

\date{}

\begin{titlepage}
\maketitle

% Abstract to be written.

\end{titlepage}

%==============================================================================
\section{Introduction}\label{s:introduction}
%==============================================================================

% TODO: Introduce the pair-instability mass gap problem and the quark
% star hypothesis.

%==============================================================================
\section{Background}\label{s:background}
%==============================================================================

% TODO: Review existing frameworks -- pair-instability supernovae,
% the predicted 50--150 M_sun gap, LIGO/Virgo observations, and the
% conventional understanding of black hole interiors.

%==============================================================================
\section{TOV Equation Analysis}\label{s:tov}
%==============================================================================

The Tolman--Oppenheimer--Volkoff equations
\citep{Tolman1939,OppenheimerVolkoff1939} govern the hydrostatic
structure of a self-gravitating, spherically symmetric body in general
relativity.  In particular, the pressure gradient equation,
%
\begin{equation}\label{eq:tov}
  \frac{dP}{dr}
  = -\frac{\bigl(\rho + P/c^{2}\bigr)
           \bigl(m + 4\pi r^{3}\,P/c^{2}\bigr)\,G}
          {r\bigl(r - 2Gm/c^{2}\bigr)}\,,
\end{equation}
%
contains terms in which the pressure~$P$ itself contributes to the
source of gravity.  As a consequence, the gravitational mass of a
compact star---the mass that determines its orbit and gravitational
field---exceeds its baryonic (rest) mass:
$M_{g} > M_{b}$.
We define $M_{g}$ as the total gravitational mass enclosed at the
stellar surface and $M_{b}$ as the integral of the baryon rest-mass
density weighted by the metric volume element.  The difference
$M_{g} - M_{b}$ is the gravitational binding energy contributed by
pressure, a quantity that grows with compactness.

The relationship between pressure, density, and composition is encoded
in the equation of state (EOS).  For neutron-star matter, we adopt
six finite-temperature nuclear EOS tables from the
\texttt{stellarcollapse.org} repository \citep{OConnorOtt2010},
evaluated on the cold, beta-equilibrated slice: LS180, LS220, and
LS375 \citep{LattimerSwesty1991}, SFHo \citep{SteinerFischerHempel2013},
DD2 \citep{TypelRopkeKlaehn2010}, and HShen \citep{Shen1998}.  These
span the range from soft (LS180, $K = 180$~MeV) to stiff (LS375,
$K = 375$~MeV) nuclear incompressibilities, producing maximum
neutron-star masses of roughly $2.0$--$3.0\;M_{\odot}$.  For
quark-star matter we use the MIT bag model \citep{Chodos1974} with bag
constant $B = 60\;\text{MeV}/\text{fm}^{3}$, which yields a maximum
mass of $\approx 2.0\;M_{\odot}$.  While pure quark stars remain
theoretical, recent Bayesian analyses of neutron-star observations
find strong evidence for quark-matter cores above
$\sim\!2\;M_{\odot}$ \citep{Annala2020}.

\begin{figure}[t]
  \centering
  \includegraphics[width=\columnwidth]{figures/tov_mass_radius.pdf}
  \caption{Gravitational mass versus radius for stable TOV
    configurations.  Solid curves show the six nuclear EOS; the dashed
    curve shows the MIT bag model quark star.  Filled circles mark the
    maximum-mass configuration for each EOS.}
  \label{fig:tov_mr}
\end{figure}

Figure~\ref{fig:tov_mr} shows the standard mass--radius diagram
obtained by integrating Eq.~\eqref{eq:tov} across a range of central
pressures for each EOS.  The neutron-star branches span radii of
$\sim\!11$--$15$~km, while the quark-star branch is more compact,
reaching only $\sim\!11$~km at maximum mass.  Both families terminate
at a maximum gravitational mass beyond which no stable equilibrium
exists.

\begin{figure}[t]
  \centering
  \includegraphics[width=\columnwidth]{figures/tov_mg_vs_mb.pdf}
  \caption{Gravitational mass versus baryonic mass for the same set of
    EOS as Fig.~\ref{fig:tov_mr}.  The dotted line is the Newtonian
    limit $M_{g} = M_{b}$.  Every TOV solution lies above this line:
    pressure contributes positively to gravitational mass.  Filled
    circles mark the maximum-mass configuration.}
  \label{fig:tov_mg_mb}
\end{figure}

The key result is shown in Figure~\ref{fig:tov_mg_mb}, which plots
$M_{g}$ against~$M_{b}$.  Every TOV solution sits above the
$M_{g} = M_{b}$ line; the departure grows with compactness.  At the
neutron-star maximum, the pressure contribution amounts to
$\sim\!20$--30\% of the baryonic mass.  The quark-star branch, being
more compact, shows a comparable or larger fractional departure at
the same~$M_{b}$.

The implications for the mass gap are direct.  When a neutron star
exceeds its TOV limit and collapses, it transitions from
neutron-degenerate to quark-degenerate matter.  Because quark matter
is more compressible, the resulting quark star is more compact and its
pressure contribution to~$M_{g}$ increases.  The baryonic mass is
conserved in the collapse, but the gravitational mass jumps upward.
The lower mass gap---the observed deficit of compact objects between
$\sim\!2.5$ and $\sim\!5\;M_{\odot}$
\citep{ThompsonKochanekAdams2019}---is the signature of this
transition: it spans the range between the highest neutron-star
$M_{g}$ and the lowest stable quark-star $M_{g}$ at the same~$M_{b}$;
the lowest stable quark-star is a black hole.

%==============================================================================
\section{Pulsar Population Analysis}\label{s:pulsars}
%==============================================================================

% TODO: Evidence that the most massive neutron stars have high spin
% rates, consistent with spin delaying collapse.

%==============================================================================
\section{BH Merger Spin Analysis}\label{s:gwtc}
%==============================================================================

% TODO: Analysis of the GWTC catalog -- Gaussian remnant spin
% distribution, spin efficiency, excess deficit, iso-m2 tracks,
% 50 M_sun formation boundary.

%==============================================================================
\section{Discussion}\label{s:discussion}
%==============================================================================

% TODO: Synthesize lines of evidence. Address M_g != M_i, the
% circularity problem with Kerr template fitting, and parsimony
% arguments.

%==============================================================================
\section{Conclusion}\label{s:conclusion}
%==============================================================================

% TODO: Summarize findings and propose future work.

%==============================================================================
\bibliography{\bib}
%==============================================================================

\end{document}
