\documentclass[letterpaper,11pt,leqno]{article}
\usepackage{paper}
\bibliographystyle{mass_gap}
\usepackage{dcolumn}
\usepackage{booktabs}
\newcolumntype{d}[1]{D{.}{.}{#1}}

% Paper title for PDF metadata:
\hypersetup{pdftitle={Quark Degeneracy and the Lower Mass Gap}}

% Path to BibTeX file:
\newcommand{\bib}{mass_gap.bib}

% Script-generated macros (re-created each time a plot script runs):
%% Auto-generated by plot_pulsar_masses.py -- do not edit
\newcommand{\pulsarCount}{37}
\newcommand{\pulsarNLow}{21}
\newcommand{\pulsarFastestLow}{2.38}
\newcommand{\pulsarSlowestLow}{2773.5}
\newcommand{\pulsarNMid}{10}
\newcommand{\pulsarFastestMid}{1.69}
\newcommand{\pulsarSlowestMid}{926.3}
\newcommand{\pulsarNHigh}{4}
\newcommand{\pulsarFastestHigh}{1.61}
\newcommand{\pulsarSlowestHigh}{39.1}
\newcommand{\pulsarNTop}{2}
\newcommand{\pulsarFastestTop}{2.56}
\newcommand{\pulsarSlowestTop}{16.8}
\newcommand{\pulsarFastestAll}{1.61}
\newcommand{\pulsarFastestSecond}{2.56}
\newcommand{\pulsarSlowestAll}{2773}
\newcommand{\pulsarSpearmanRho}{-0.41}
\newcommand{\pulsarSpearmanP}{1.1 \times 10^{-2}}

%% Auto-generated by plot_pulsar_population.py -- do not edit
\newcommand{\popTotal}{4286}
\newcommand{\popNormal}{3305}
\newcommand{\popBinRangeA}{100--300\,ms}
\newcommand{\popCountA}{545}
\newcommand{\popDensityA}{1142}
\newcommand{\popBinRangeB}{300--1000\,ms}
\newcommand{\popCountB}{1660}
\newcommand{\popDensityB}{3175}
\newcommand{\popBinRangeC}{1000--3000\,ms}
\newcommand{\popCountC}{929}
\newcommand{\popDensityC}{1947}
\newcommand{\popBinRangeD}{3000--10000\,ms}
\newcommand{\popCountD}{153}
\newcommand{\popDensityD}{293}
\newcommand{\popDeficitAB}{0}
\newcommand{\popSigmaAB}{0.0}
\newcommand{\popDeficitBC}{586}
\newcommand{\popSigmaBC}{16.2}
\newcommand{\popDeficitCD}{865}
\newcommand{\popSigmaCD}{33.9}
\newcommand{\popLinearBinWidth}{100}
\newcommand{\popLinearNBins}{15}

%% Auto-generated by plot_bh_merger_spin.py -- do not edit
\newcommand{\gwNbbh}{157}
\newcommand{\gwNdeficit}{154}
\newcommand{\gwMedianExcess}{0.0029}
\newcommand{\gwFracPositive}{84\%}
\newcommand{\gwMedianFracDeficit}{0.046}
\newcommand{\gwAlphaQS}{0.0083}
\newcommand{\gwMedianChiEff}{0.05}
\newcommand{\gwRhoExcessChi}{+0.433}
\newcommand{\gwPExcessChi}{2.1 \times 10^{-8}}
\newcommand{\gwRhoExcessChif}{+0.506}
\newcommand{\gwPExcessChif}{2.2 \times 10^{-11}}
\newcommand{\gwRhoDeficitQ}{-0.510}
\newcommand{\gwPDeficitQ}{1.5 \times 10^{-11}}
\newcommand{\gwRhoExcessPerMtwoQ}{-0.262}
\newcommand{\gwPExcessPerMtwoQ}{0.0010}
\newcommand{\gwRhoExcessQ}{-0.194}
\newcommand{\gwPExcessQ}{0.0159}


\begin{document}

\title{Quark Degeneracy and the Lower Mass Gap}

\author{Gregory L. Frazier \thanks{Independent researcher. The author
    thanks Claude (Anthropic) for assistance with analysis and
    manuscript preparation.}}

\date{}

\begin{titlepage}
\maketitle

\begin{abstract}
The observed deficit of compact objects between ${\sim}2.5$ and
${\sim}5\;M_{\odot}$---the lower mass gap---has no widely accepted
physical explanation.  We propose that the gap is the signature of the
phase transition from neutron-degenerate to quark-degenerate matter,
implying that black holes possess internal structure rather than
singularities.  Four independent lines of evidence support this
interpretation: the TOV equations anticipate the gap at the
neutron--quark transition; massive pulsars disappear as they spin
down, consistent with centrifugal support delaying collapse; failed
supernovae confirm that cores well below $5\;M_{\odot}$ collapse to
form black holes; and LIGO black hole merger remnants exhibit spin-dependent mass
deficits consistent with structured matter.
\end{abstract}

\end{titlepage}

%==============================================================================
\section{Introduction}\label{s:introduction}
%==============================================================================

Gravitational-wave and X-ray observations reveal a gap in the mass
spectrum of compact objects. Neutron stars with dynamically measured
masses cluster below ${\sim}2.5;M_{\odot}$ \citep{LattimerNSmasses},
while black holes in X-ray binaries begin near ${\sim}5;M_{\odot}$
\citep{Bailyn1998}.  The deficit of objects between these limits is
the \emph{lower mass gap}.

The reality and origin of the gap have been debated for two decades.
\citet{Bailyn1998} first identified the clustering of X-ray binary
black holes above ${\sim}5\;M_{\odot}$; \citet{Ozel2010} argued on
statistical grounds that the gap is real and not a selection effect,
and \citet{Farr2011} reached the same conclusion with a Bayesian
analysis.  \citet{Kreidberg2012} countered that systematic errors in
X-ray binary mass measurements could partially fill the gap.  On the
theoretical side, \citet{Fryer2012} showed that rapid supernova
explosion mechanisms can produce a gap by ejecting the envelope before
significant fallback, while delayed mechanisms fill it in.  More
recently, the LIGO--Virgo detection of GW190814---a merger whose
$2.6\;M_{\odot}$ secondary sits squarely in the gap
\citep{GW190814}---and population synthesis studies
\citep{Zevin2020} have renewed the question of whether the gap is
truly empty or merely underpopulated.  No consensus has emerged on the
physical origin of the deficit.

We propose that the lower mass gap is the signature of a phase
transition; when a neutron star exceeds the Tolman-Oppenheimer-Volkoff
(TOV) stability limit, its matter collapses from neutron degeneracy to
quark degeneracy, producing a step-function increase in compactness,
pressure, and thus gravitational mass. But the TOV equations not only
predict the increase in gravitational mass---they also predict that,
at the point where a neutron star collapses, the result is a black
hole.  The pressure contribution to gravitational mass cannot operate
if the mass exists as a dimensionless point.  As such, this
explanation requires that black holes are not singularities but
structured bodies---objects with a real equation of state, finite
volume, and internal pressure inside the event horizon.  This is a
testable requirement; if black holes have internal structure, their
mergers should exhibit signatures inconsistent with point-mass
predictions.  Thus we present four lines of evidence:
% one
TOV equations predict the gap (\S\ref{s:tov});
% two
massive pulsars (\S\ref{ss:pulsar-masses}) and the pulsar population
(\S\ref{ss:pulsar-population}) show the expected spin-dependent
disappearance;
% three
failed supernovae independently confirm sub-$5\;M_{\odot}$ collapse of
objects to black holes (\S\ref{ss:rsg});
% four
and LIGO merger remnants carry spin signatures consistent with
structured matter, in contrast to point singularities (\S\ref{s:gwtc}).

%==============================================================================
\section{TOV Equation Analysis}\label{s:tov}
%==============================================================================

The Tolman--Oppenheimer--Volkoff equations
\citep{Tolman1939,OppenheimerVolkoff1939} govern the hydrostatic
structure of a self-gravitating, spherically symmetric body in general
relativity.  In particular, the pressure gradient equation,
%
\begin{equation}\label{eq:tov}
  \frac{dP}{dr}
  = -\frac{\bigl(\rho + P/c^{2}\bigr)
           \bigl(m + 4\pi r^{3}\,P/c^{2}\bigr)\,G}
          {r\bigl(r - 2Gm/c^{2}\bigr)}\,,
\end{equation}
%
contains terms in which the pressure~$P$ itself contributes to the
source of gravity.  As a consequence, the gravitational mass of a
compact star---the mass that determines its orbit and gravitational
field---exceeds its baryonic (rest) mass:
$M_{g} > M_{b}$.
We define $M_{g}$ as the total gravitational mass enclosed at the
stellar surface.  The baryonic mass~$M_{b}$ is defined so as to
isolate the ground-state structural energy of each matter type,
excluding the internal kinetic energy that generates pressure.
For neutron-star matter, $M_{b}$ is the integral of the baryon
rest-mass density~$\rho_{0}$ over coordinate volume---the mass the
baryons would have if dispersed to infinity with zero internal energy.
For quark-star matter, $M_{b}$ is the integral of the bag vacuum
energy density~$4B/c^{2}$ over coordinate volume---the QCD vacuum cost
of creating the quark-matter volume, with no quark kinetic
contribution.  Both definitions strip the internal energy that
generates pressure and keep only the structural ground state.  The
difference $M_{g} - M_{b}$ then measures pressure's contribution to
the gravitating mass for both matter types, and grows with
compactness.

The relationship between pressure, density, and composition is encoded
in the equation of state (EOS).  For neutron-star matter, we adopt
six finite-temperature nuclear EOS tables from the
\texttt{stellarcollapse.org} repository \citep{OConnorOtt2010},
evaluated on the cold, beta-equilibrated slice: LS180, LS220, and
LS375 \citep{LattimerSwesty1991}, SFHo \citep{SteinerFischerHempel2013},
DD2 \citep{TypelRopkeKlaehn2010}, and HShen \citep{Shen1998}.  These
span the range from soft (LS180, $K = 180$~MeV) to stiff (LS375,
$K = 375$~MeV) nuclear incompressibilities, producing maximum
neutron-star masses of roughly $2.0$--$3.0\;M_{\odot}$.  For
quark-star matter we use the MIT bag model \citep{Chodos1974} with
three representative bag constants, $B = 60$, $100$, and
$160\;\text{MeV}/\text{fm}^{3}$, spanning the range from soft to stiff
quark matter.  These produce maximum gravitational masses of roughly
$2.0$, $1.6$, and $1.2\;M_{\odot}$, respectively.  While pure quark
stars remain theoretical, recent Bayesian analyses of neutron-star
observations find strong evidence for quark-matter cores above
${\sim}2\;M_{\odot}$ \citep{Annala2020}.

%% \begin{figure}[t]
%%   \centering
%%   \includegraphics[width=\columnwidth]{figures/tov_mass_radius.png}
%%   \caption{Gravitational Mass vs Radius, Neutron and Quark Stars}
%%   \note{Stable configurations. Solid curves are the six nuclear EOS; the
%%     dashed curves are MIT bag model quark stars with
%%     $B = 60$, $100$, and $160\;\text{MeV}/\text{fm}^{3}$.  Filled
%%     circles mark the maximum-mass configuration for each EOS.
%%   }
%%   \label{fig:tov_mr}
%% \end{figure}

\begin{figure}[t]
\subcaptionbox{Gravitational Mass vs. Radius\label{fig:tov_mr}}{
  \includegraphics[scale=0.4]{figures/tov_mass_radius.png}
}\hfill
\subcaptionbox{Gravitational Mass vs. Baryonic Mass\label{fig:tov_mg_mb}}{
  \includegraphics[scale=0.4]{figures/tov_mg_vs_mb.png}
}
\caption{Neutron and Quark Star Stable Configurations}
\note{Solid curves are the six nuclear EOS; the
    dashed curves are MIT bag model quark stars with
    $B = 60$, $100$, and $160\;\text{MeV}/\text{fm}^{3}$. For a given
    baryonic mass $M_b$, quark stars have greater gravitational mass
    $M_g$ than do neutron stars.
}
\label{fig:tov_figs}
\end{figure}


Figure~\ref{fig:tov_mr} shows the standard mass--radius diagram
obtained by integrating Eq.~\eqref{eq:tov} across a range of central
pressures for each EOS.  The neutron-star branches span radii of
${\sim}11$--$15$~km, while the quark-star branches are more compact,
reaching ${\sim}7$--$11$~km at maximum mass depending on~$B$.  Both
families terminate at a maximum gravitational mass beyond which no
stable equilibrium exists.

%% \begin{figure}[t]
%%   \centering
%%   \includegraphics[width=\columnwidth]{figures/tov_mg_vs_mb.png}
%%   \caption{Gravitational Mass $M_g$ vs. Baryonic Mass $M_b$}
%%   \note{Same set of
%%     EOS as Fig.~\ref{fig:tov_mr}.  The dotted line is the Newtonian
%%     limit $M_{g} = M_{b}$.  Neutron-star curves (solid) lie just above
%%     this line, with pressure contributing ${\sim}10$--15\% of~$M_{g}$.
%%     Quark-star curves (dashed) show dramatically larger separation;
%%     their baryonic mass is defined via the bag vacuum energy~$4B/c^{2}$
%%     alone, so that the ultrarelativistic quark kinetic energy
%%     (which generates pressure) appears entirely in $M_{g} - M_{b}$.
%%     Filled circles mark the maximum-mass configuration.}
%%   \label{fig:tov_mg_mb}
%% \end{figure}

The key result is shown in Figure~\ref{fig:tov_mg_mb}, which plots
$M_{g}$ against~$M_{b}$.  Every TOV solution sits above the
$M_{g} = M_{b}$ line; the departure grows with compactness.  The
neutron-star curves (solid) hug the diagonal, with the pressure
contribution amounting to ${\sim}10$--15\% of~$M_{g}$ at maximum mass.
The quark-star curves (dashed) depart dramatically: because the
baryonic mass counts only the bag vacuum energy ($4B/c^{2}$), the
entire ultrarelativistic quark kinetic energy ($3P/c^{2}$) appears as
excess gravitational mass.  At the quark-star maximum,
$M_{g}/M_{b} \sim 1.5$--2, meaning that pressure-generated energy
accounts for roughly half the gravitating mass.

The implications for the mass gap are direct.  When a neutron star
exceeds its TOV limit, it collapses from neutron-degenerate to
quark-degenerate matter.  The baryonic mass is conserved, but because
the resulting quark star is more compact, its pressure contribution
to~$M_{g}$ is larger: the gravitational mass jumps upward.  The lower
mass gap---the observed deficit of compact objects between ${\sim}2.5$
and ${\sim}5\;M_{\odot}$ \citep{ThompsonKochanekAdams2019}---is the
signature of this transition, spanning from the pre-collapse
neutron-star~$M_{g}$ to the post-collapse quark-star~$M_{g}$.
Crucially, the post-collapse gravitational mass exceeds the maximum
stable quark-star solution: the object is not a quark star but a black
hole. Yet, the smallest observed black holes are over
5\;$\text{M}_\odot$, suggesting that black holes have a pressure
component to their $M_g$, which in turn implies that they are coherent
stellar objects---they are quark stars.

The spread of EOS and bag constants in Fig.~\ref{fig:tov_figs}
reflects current uncertainty in dense-matter physics; the precise
behavior of matter at these densities is not yet known.
Nonetheless, the qualitative picture is robust.  Both families show
a steepening $M_{g}$--$M_{b}$ relationship with increasing baryonic
mass, and quark stars consistently exhibit a far larger pressure
contribution to~$M_{g}$ than neutron stars at comparable~$M_{b}$.

This pressure contribution to~$M_{g}$ is not merely a theoretical
refinement; it is an observable that distinguishes structured objects
from singularities.  A point singularity has zero volume, no equation
of state, and no internal pressure---the TOV mechanism cannot operate.
If black holes are singularities, their gravitational mass equals
their baryonic mass by construction.  The fact that the smallest
observed black holes sit well above the neutron star maximum implies a
pressure contribution, and therefore internal structure.

That said, the TOV analysis demonstrates only that a phase
transition \emph{would} produce a mass gap whose endpoint is a
black hole; it does not, by itself, show that neutron stars
\emph{do} collapse.  Indeed, the absence of a stable post-collapse
solution might be read as evidence against collapse.
Observational evidence that collapse is occurring is presented in
the following section.

%==============================================================================
\section{Observational Evidence for Collapse}\label{s:collapse}
%==============================================================================

The TOV analysis predicts that neutron stars exceeding the stability
limit collapse through the mass gap into quark-degenerate
configurations.  This section presents three independent lines of
evidence that such collapses are occurring: the spin-dependent
disappearance of massive pulsars, the thinning of the pulsar
population at long spin periods, and the failed-supernova phenomenon
in red supergiants.

\subsection{Pulsars with Known Mass}\label{ss:pulsar-masses}

We assemble a sample of \pulsarCount{} pulsars with dynamically measured masses
from the \texttt{stellarcollapse.org} compilation
\citep{OConnorOtt2010}, cross-matched with spin periods from the ATNF
Pulsar Catalogue \citep{Manchester2005}.  The sample spans
$1.18$--$2.74\;M_{\odot}$ and includes four binary categories: NS--WD
(21 systems), NS--NS (7), X-Ray/Optical (2), and NS--MS (3).  Three
of the most precisely measured high-mass pulsars---J0740+6620
($2.08 \pm 0.07\;M_{\odot}$; \citealt{Cromartie2020}), J0348+0432
($2.01 \pm 0.04\;M_{\odot}$), and J1614$-$2230
($1.928 \pm 0.017\;M_{\odot}$)---anchor the high end of the
distribution.

\begin{figure}[t]
  \centering
  \includegraphics[width=\columnwidth]{figures/pulsar_mass_period.png}
  \caption{Mass vs.\ Spin Period, Pulsars with Known Mass}
  \note{Examined \pulsarCount{} pulsars with dynamical mass
    measurements.  (a)~Scatter plot of 37 pulsars with known mass. No
    pulsar above 1.6 $\text{M}_\odot$ has a slower period than
    80~ns. (b)~Period range per mass-bin: horizontal segments span
    from the fastest to the slowest period in each bin, with sample
    size annotated.  The fastest periods are comparable across all
    bins (\pulsarFastestAll--\pulsarFastestSecond~ms), but the slowest
    period drops from \pulsarSlowestAll~ms to \pulsarSlowestTop~ms as
    mass increases. Note also that the number of pulsars in each bin
    decreases as the masses increase.}
  \label{fig:pulsar_mass_period}
\end{figure}

Figure~\ref{fig:pulsar_mass_period} reveals the key pattern.  When the
sample is binned by mass, the fastest spin period in each bin is
comparable (\pulsarFastestAll--\pulsarFastestSecond~ms), showing that
millisecond pulsars exist at all masses.  But the slowest period
shrinks dramatically: from \pulsarSlowestLow~ms in the
$1.0$--$1.5\;M_{\odot}$ bin down to \pulsarSlowestTop~ms above
$2.5\;M_{\odot}$.  Massive pulsars are found exclusively at fast spin
rates; no slowly rotating high-mass neutron star has been observed.

This pattern is consistent with spin providing centrifugal support
that delays collapse past the TOV stability limit.  As pulsars spin
down, those with the highest baryonic
mass cross the stability boundary first, disappearing from the
observable population.  Only those massive pulsars that still retain
fast spin remain visible.  A Spearman rank correlation confirms the
trend: $\rho = \pulsarSpearmanRho$ ($p = \pulsarSpearmanP$),
indicating a statistically significant negative association between
mass and spin period.

That said, these \pulsarCount{} pulsars are the subset with
dynamically measured masses---a small, heterogeneous sample.  The
next subsection uses the full ATNF pulsar catalogue to test this
interpretation with stronger statistics.

\subsection{Pulsar Population Analysis}\label{ss:pulsar-population}

The preceding analysis showed that among pulsars with measured masses,
the most massive are found exclusively at fast spin rates.  If larger
pulsars collapse as they spin down, the effect should leave a clear
imprint on the period distribution of the full pulsar population.

We draw \popTotal{} pulsars from the ATNF Pulsar Catalogue
\citep{Manchester2005} and restrict the sample to the \popNormal{}
normal (non-recycled) pulsars with $P \geq 100$~ms.  Under magnetic
dipole braking ($P\dot{P} \approx \text{const}$), the period
derivative $\dot{P} \propto 1/P$, so pulsars traverse a given period
interval more slowly at long periods.  The dwell time in an interval
$dP$ scales as $P\,dP$, and in a steady-state population with
constant birth rate the number density per log-decade of period should
be non-decreasing.

\begin{figure}[t]
\subcaptionbox{Density per log-decade\label{fig:pop_logbins}}{%
  \includegraphics[scale=0.4]{figures/pulsar_population_logbins.png}%
}\hfill
\subcaptionbox{Fixed-width bins vs.\ model\label{fig:pop_linear}}{%
  \includegraphics[scale=0.4]{figures/pulsar_population_linear.png}%
}
\caption{Period Distribution of Normal Pulsars ($P \geq 100$~ms)}
\note{(a)~Pulsars per log-decade in four logarithmically spaced bins.
  Blue bars are consistent with a non-decreasing density; red bars
  fall below the peak, with Poisson significance annotated.
  (b)~Observed counts in \popLinearBinWidth{}~ms bins (blue) compared
  with the steady-state dipole spin-down prediction (count $\propto P$,
  red step line, normalised at 200--300~ms).  Note the logarithmic
  vertical axis for (b).}
\label{fig:pop_figs}
\end{figure}

Figure~\ref{fig:pop_logbins} tests this prediction.  The density
rises from \popDensityA{} pulsars per decade in the \popBinRangeA{}
bin to a peak of \popDensityB{} in the \popBinRangeB{} bin, then
drops precipitously: \popDensityC{} in the \popBinRangeC{} bin (a
\popSigmaBC{}$\sigma$ deficit relative to the non-decreasing null)
and just \popDensityD{} in the \popBinRangeD{} bin
(\popSigmaCD{}$\sigma$).  Figure~\ref{fig:pop_linear} shows the same
data in fixed-width bins alongside the steady-state prediction.  The
observed counts fall exponentially while the model rises linearly; by
3~s the gap exceeds an order of magnitude.

The conventional explanation for the missing long-period pulsars is
luminosity death: the cessation of coherent radio emission below a
critical spin-down power \citep{FaucherGiguereKaspi2006}.  However,
the death line operates at $P \gtrsim 5$~s for typical magnetic fields
and so does not account for a population decline that is already
observable at 0.1~s and is distributed across the entire range of
pulsar spin periods.  Population synthesis studies have invoked
magnetic field decay to fill the gap, but the required timescales are
uncertain and \citet{FaucherGiguereKaspi2006} found no evidence for
significant field decay over pulsar lifetimes.  Most recently,
\citet{Sautron2024} showed that a death line is not necessary to
reproduce the observed $P$--$\dot{P}$ diagram: their simulations with
and without a death line are statistically indistinguishable once the
maximum pulsar age is capped at ${\sim}4 \times 10^{7}$~yr.  What is
required is simply that old pulsars are absent---but the physical
reason for their absence remains unexplained in that framework.

The collapse hypothesis of \S\ref{ss:pulsar-masses} offers a natural
explanation for observed population dynamics.  As pulsars spin down,
those with baryonic masses near the TOV limit lose centrifugal support
and collapse to quark stars (black holes), physically removing them
from the observable population.  This predicts a gradual thinning of
the population that begins well before the nominal death
line---precisely the pattern seen in Figures~\ref{fig:pop_logbins}
and~\ref{fig:pop_linear}.  It further predicts that the deficit should
be strongest among the most massive pulsars, a test that targeted mass
measurements at long periods could address.

\subsection{The Red Supergiant Problem}\label{ss:rsg}

The preceding subsections established that pulsars are disappearing
from the observable population as they spin down---evidence that
compact objects with baryonic mass well below $5\;M_{\odot}$ are
collapsing to form black holes.  Here we examine a completely
independent line of evidence that points to the same conclusion.

Red supergiants (RSGs) are massive evolved stars in the final stages
of nuclear burning, with initial (zero-age main-sequence) masses of
roughly $8$--$25\;M_{\odot}$.  Stars in this range are expected to
end their lives as core-collapse supernovae.  However,
\citet{Smartt2009} analysed pre-explosion images of a volume-limited
sample of Type~II-P supernovae and found that no progenitor more
massive than $16.5 \pm 1.5\;M_{\odot}$ had been identified---despite
RSGs with masses up to ${\sim}25\;M_{\odot}$ being well documented
in the Local Group.  An expanded sample of 45 progenitors confirmed
the deficit: 13 high-luminosity progenitors were expected under a
Salpeter initial mass function, but only one was found
\citep{Smartt2015}.  This discrepancy---the absence of supernova
progenitors in the ${\sim}17$--$25\;M_{\odot}$ range---is known as
the \emph{red supergiant problem}.

The natural explanation is that the most massive RSGs do not produce
visible supernovae: their cores collapse directly to black holes,
with the stellar envelope falling back rather than being ejected.
\citet{Kochanek2020} showed that this ``failed supernova'' hypothesis
provides a consistent explanation for both the missing progenitors
and the observed black hole mass function.  A systematic search by
the Large Binocular Telescope survey, monitoring luminous stars in
27~nearby galaxies over 11~years, estimates that
$16^{+23}_{-12}$\% of core-collapse events produce failed supernovae
\citep{NeustadtKochanek2021}.

Two candidate failed supernovae have now been identified
observationally.  NGC~6946-BH1, a ${\sim}25\;M_{\odot}$ red
supergiant, underwent a brief optical outburst in 2009 and then
vanished; by 2015 it was undetectable in optical light
\citep{Adams2017}, though subsequent JWST imaging revealed that the
pre-disappearance source was a blend of at least three objects,
complicating the interpretation \citep{Beasor2024}.  The clearer
case is M31-2014-DS1, a hydrogen-depleted supergiant in the
Andromeda Galaxy \citep{De2026}.  With an initial mass of
${\sim}13\;M_{\odot}$---notably below the RSG problem threshold,
extending the failed-supernova phenomenon to lower masses---and a
terminal mass of ${\sim}5\;M_{\odot}$, this star faded by a factor
of $\gtrsim\!10^{4}$ in optical light between 2017 and 2022 with no
associated supernova.  Modelling indicates that ${\sim}98$\% of the
stellar mass collapsed or fell back, forming a
${\sim}5\;M_{\odot}$ black hole.

The implications for the mass gap are direct.  Stellar evolution
models show that the iron core mass at the point of collapse is
${\sim}1.3$--$1.8\;M_{\odot}$ across the entire range from 8 to
$25\;M_{\odot}$ \citep{WoosleyHegerWeaver2002, Sukhbold2016}.
This is the mass that actually undergoes gravitational collapse;
subsequent accretion of envelope material occurs only after the
compact remnant has formed.  Even the helium core---an upper bound
on the material that can plausibly fall back---is only
${\sim}4$--$9\;M_{\odot}$ for stars of $15$--$25\;M_{\odot}$.
In every case, the baryonic mass of the collapsing core is
certainly less than $5\;M_{\odot}$ and most likely less than
$2\;M_{\odot}$.

The failed-supernova evidence therefore establishes, independently
of the pulsar analysis, that cores with baryonic mass well below
$5\;M_{\odot}$---most likely ${\sim}1.5\;M_{\odot}$---can and do
collapse to form black holes.  This is precisely the process
proposed in \S\ref{ss:pulsar-masses} and \S\ref{ss:pulsar-population}
for pulsars that lose centrifugal support as they spin down.  That
the same phenomenon---black hole formation from a
${\sim}2\;M_{\odot}$ compact core---appears in two completely
unrelated astrophysical settings strengthens the case that the
lower mass gap is the signature of this collapse, as predicted by
the TOV analysis of \S\ref{s:tov}.

%==============================================================================
\section{BH Merger Spin Analysis}\label{s:gwtc}
%==============================================================================

The preceding sections established that compact objects with baryonic
mass well below $5\;M_{\odot}$ collapse to form black holes, and that
the TOV equations predict a mass gap at the neutron--quark transition.
But those arguments do not, by themselves, demonstrate that black holes
contain structured matter.  If black holes are quark stars---compact
objects with a real equation of state, finite radius, and internal
pressure---then their mergers should leave observable signatures in the
gravitational-wave record that differ from the predictions of the
point-singularity model.  The GWTC catalog \citep{GWTC3,GWOSC2023}
provides exactly the data needed to test this.

Consider two models of what happens when two black holes merge.  In the
singularity model, each black hole is a point mass enclosed by an event
horizon; the mass deficit $m_1 + m_2 - m_{\rm final}$ is energy
radiated as gravitational waves.  The ``spin'' of a Kerr black hole is
a property of the spacetime geometry: angular momentum is defined as a
conserved charge at spatial infinity, and the singularity itself has no
moment of inertia and no internal structure for centrifugal force to act
upon.  In the quark star model, two physical objects collide and merge.
The remnant is a spinning body with a moment of inertia, and its spin
provides centrifugal support that reduces the internal pressure and
therefore reduces the gravitational mass~$M_g$ relative to the conserved
baryonic mass~$M_b$.  No baryonic mass is converted to energy---the
gravitational-wave energy comes from the rotational kinetic energy of
the inspiral---and the apparent mass deficit includes a spin-induced
reduction in~$M_g$.

The two models also differ in how they account for the gravitational
waves observed after the merger (the ``ringdown'').  In the singularity
model, the inspiral GWs are driven radiation, sourced by the
stress-energy of orbiting masses; but the post-merger GWs are
quasi-normal modes---source-free oscillations of the Kerr geometry
ringing down to equilibrium.  The radiation mechanism changes at
merger from matter-sourced to geometry-sourced.  In the quark star
model, a single mechanism operates throughout: the time-varying
quadrupole moment of a physical mass distribution, first from two
bodies in mutual orbit, then from a distorted remnant settling into
axial symmetry.  Both models predict post-merger gravitational
waves---quasi-normal modes have been a prediction of the Kerr model
since the 1970s---but they disagree on what is radiating.

Both models predict that the mass deficit depends on the symmetric mass
ratio $\eta = m_1 m_2 / (m_1 + m_2)^{2}$, because the orbital dynamics
that govern both GW radiation and angular momentum transfer are
controlled by~$\eta$.  We use the non-spinning GR prediction of
\citet{BuonannoKidderLehner2008} as a baseline:
%
\begin{equation}\label{eq:frac_gr}
  f_{\rm GR} = 0.0572\,\eta + 0.498\,\eta^{2}\,.
\end{equation}
%
This represents the minimum deficit predicted by the singularity model
(non-spinning progenitors).  The quark star model predicts additional
deficit from spin-induced $M_g$ reduction, proportional to the remnant
spin generated.

We test this using \gwNbbh{} confident binary black hole events
($m_1 > 3\;M_{\odot}$, $m_2 > 3\;M_{\odot}$,
$p_{\rm astro} > 0.5$) from the GWTC catalog, of which \gwNdeficit{}
have both a measured final mass and effective spin $\chi_{\rm eff}$.
For each event we compute the remnant spin~$\chi_f$ from the
numerical-relativity fitting formula
%
\begin{equation}\label{eq:chi_f}
  \chi_f = L_{\rm orb}(\eta) + \frac{\chi_{\rm eff}\,(1+q^{2})}{(1+q)^{2}}\,,
\end{equation}
%
where $L_{\rm orb}(\eta) = 2\sqrt{3}\,\eta - 3.52\,\eta^{2}
+ 2.58\,\eta^{3}$ is the orbital angular momentum contribution
and the second term is the spin angular momentum contribution
(assuming equal component spins $a_1 = a_2 = \chi_{\rm eff}$).
The excess fractional deficit is then
$\Delta f = f_{\rm obs} - f_{\rm GR}$, where
$f_{\rm obs} = (m_1 + m_2 - m_{\rm final}) / (m_1 + m_2)$.

\paragraph{Test 1: excess deficit vs.\ remnant spin}

Figure~\ref{fig:excess_vs_chif} plots the excess fractional deficit
$\Delta f$ against the computed remnant spin~$\chi_f$ for all
\gwNdeficit{} events.  A Spearman rank correlation gives
$\rho = \gwRhoExcessChif$ ($p = \gwPExcessChif$), a highly significant
positive association.  The median excess is $\gwMedianExcess$, and
\gwFracPositive{} of events exceed the non-spinning GR prediction.
The binned medians trace a clear rising curve from near zero at
$\chi_f \approx 0.55$ to ${\sim}0.012$ at $\chi_f \approx 0.85$.

The quark star model offers a direct physical mechanism for this
correlation: a spinning body with internal structure experiences
centrifugal support that reduces its central pressure and therefore
its gravitational mass.  More remnant spin means more centrifugal
support and a larger apparent mass deficit.  In the singularity model,
the remnant's ``spin'' is a geometric property of the Kerr metric, with
no internal structure for centrifugal force to act upon; the connection
between spin and mass deficit is less direct.

\begin{figure}[t]
  \centering
  \includegraphics[width=\columnwidth]{figures/bh_merger_excess_vs_chif.png}
  \caption{Excess Fractional Mass Deficit vs.\ Remnant Spin}
  \note{Each point is a GWTC binary black hole merger.  The excess
    is computed relative to the non-spinning GR prediction
    (Eq.~\ref{eq:frac_gr}).  Points are coloured by primary mass
    $m_1$.  Remnant spin $\chi_f$ is computed from
    Eq.~\eqref{eq:chi_f}.  Red squares show binned medians.
    The Spearman correlation ($\rho = \gwRhoExcessChif$,
    $p = \gwPExcessChif$) confirms that higher remnant spin is
    associated with larger excess deficit.}
  \label{fig:excess_vs_chif}
\end{figure}

\paragraph{Test 2: deficit per unit $m_2$ vs.\ mass ratio}

Figure~\ref{fig:deficit_per_m2} plots the total mass deficit per unit
$m_2$ against the mass ratio~$q$.  Two prediction curves are shown.
The singularity model curve (red, dashed) uses the non-spinning GR
formula (Eq.~\ref{eq:frac_gr}).  The quark star model curve (purple,
solid) adds a spin-induced $M_g$ reduction proportional to the square
of the orbital angular momentum deposited:
$f_{\rm QS} = f_{\rm GR} + \alpha\,L_{\rm orb}(\eta)^{2}$,
where $\alpha = \gwAlphaQS$ is fit from the excess-vs-$\chi_f$ data
of Test~1.  Both curves are normalized by~$m_2$ to measure the deficit
generated per solar mass of secondary, allowing comparison across
mergers of different mass scales.

The two curves have the same shape because both effects---GW radiation
and spin deposition---are governed by the same orbital dynamics through
$\eta$.  Equal-mass mergers maximize both GW emission and angular
momentum transfer; lopsided mergers reduce both.  The quark star model
shifts the curve upward by the spin-induced contribution, which is
nearly constant in deficit/$m_2$ across all mass ratios because the
decreasing $L_{\rm orb}^{2}$ at low~$q$ is offset by the increasing
$1/m_2$ amplification.

The binned medians (green squares) track the quark star model curve
across the full range of mass ratios, sitting consistently above the
singularity model prediction.  A Spearman correlation of the raw
deficit/$m_2$ against $q$ gives $\rho = \gwRhoDeficitQ$
($p = \gwPDeficitQ$).

\begin{figure}[t]
  \centering
  \includegraphics[width=\columnwidth]{figures/bh_merger_deficit_per_m2.png}
  \caption{Mass Deficit per Unit $m_2$ vs.\ Mass Ratio}
  \note{Blue points: observed deficit divided by $m_2$ for each GWTC
    event.  Red dashed curve: singularity model (non-spinning GR,
    Eq.~\ref{eq:frac_gr}).  Purple solid curve: quark star model,
    adding spin-induced $M_g$ reduction with $\alpha = \gwAlphaQS$
    fit from Test~1.  Green squares: binned medians.  The data
    follow the quark star prediction, not the singularity prediction
    ($\rho = \gwRhoDeficitQ$, $p = \gwPDeficitQ$).}
  \label{fig:deficit_per_m2}
\end{figure}

Both tests point in the same direction.  The mass deficit in binary
black hole mergers correlates strongly with the remnant spin
(Test~1), and the observed deficit level matches the quark star
prediction across all mass ratios while consistently exceeding the
singularity prediction (Test~2).  The quark star model explains both
results with a single physical mechanism---centrifugal support reducing
$M_g$---using one free parameter ($\alpha$) fit from Test~1 and
validated against Test~2.

A caveat is in order.  The non-spinning GR formula
(Eq.~\ref{eq:frac_gr}) is a conservative baseline: the singularity
model with spinning progenitors also predicts more deficit than the
non-spinning case, because spin-orbit coupling enhances GW emission.
Quantifying the spinning-GR prediction requires full numerical
relativity fitting formulas that include component spins, which is
beyond the scope of this analysis.  The present comparison establishes
that the data depart from the non-spinning baseline in the direction
and by the amount that the quark star model predicts; whether the
spinning singularity model can account for the same departure is a
question for future work.

%==============================================================================
\section{Conclusion}\label{s:conclusion}
%==============================================================================

The lower mass gap between neutron stars and black holes has lacked a
physical explanation.  We have argued that it is the signature of a
phase transition from neutron degeneracy to quark degeneracy, driven
by the same pressure--gravity coupling embedded in the TOV equations.
A direct implication is that black holes are not singularities but
structured stellar objects with finite volume, an equation of state,
and internal pressure that contributes to their gravitational mass.

Four independent lines of evidence support this picture.  The TOV
equations show that a neutron-to-quark transition produces a
gravitational-mass jump that anticipates the observed gap
(\S\ref{s:tov}).  Massive pulsars are found exclusively at fast spin
rates, consistent with centrifugal support delaying collapse past the
stability limit (\S\ref{ss:pulsar-masses}).  An assessment of the full
observed pulsar population, for which mass information is unavailable,
shows that pulsars are disappearing from the population, consistant
with centrifugal support delaying collapse
(\S\ref{ss:pulsar-population}).  Failed supernovae confirm,
independently of the pulsar evidence, that stellar cores with baryonic
mass well below $5\;M_{\odot}$ collapse to form black holes
(\S\ref{ss:rsg}).  Finally, LIGO merger remnants show a spin-dependent
mass deficit that correlates with remnant spin and exceeds the
non-spinning GR prediction by the amount a structured-matter model
predicts---with a single free parameter calibrated on one observable
and independently validated on a second (\S\ref{s:gwtc}).

A limitation of this analysis is that all LIGO parameters are
extracted from waveform templates built on the Kerr metric, which
assumes the singularity model.  Using these parameters to test a
non-singularity hypothesis introduces a circularity that cannot be
fully resolved with current data.  Nonetheless, the consistency of the
quark star model across four independent lines of evidence---two of
which (pulsars and failed supernovae) involve no gravitational-wave
data at all---suggests that the result is not an artifact of template
assumptions.

Several directions for future work are apparent.  Comparing the quark
star prediction against the full spinning-Kerr GR baseline, using
numerical relativity fitting formulas with component spins, would
sharpen the distinction between the two models.  Targeted mass
measurements of long-period pulsars could test the prediction that the
spin-down disappearance is strongest among the most massive neutron
stars.  And the continued growth of the gravitational-wave catalog will
improve the statistical power of the merger analysis presented here.

%==============================================================================
\bibliography{\bib}
%==============================================================================

\end{document}
